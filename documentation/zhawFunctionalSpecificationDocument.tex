%%	build-queue:
%%	
%%	¦¦¦ very first run. No .ist files yet
%%	¦¦	index, citation/bibliography or glossary changed
%%	¦	every change apart from the above mentioned. Double run for labels and toc.
%%	
%%	¦¦¦xelatex
%%	¦¦makeglossaries
%%	¦¦makeindex
%%	¦¦bibtex
%%	¦xelatex
%%	¦xelatex
%%	

\RequirePackage[l2tabu,orthodox]{nag}
\documentclass[10pt,a4paper,titlepage,twoside,german]{zhawreprt}

\include{packages}
\if false
%\newcommand{\FullGlossaryEntry}[7]{%Abkz., Typ, Name, Name(plural), Beschr., Beschr.(plural), Erkl.
%\newglossaryentry{#1}{type=\acronymtype,
%name={\textit{#3}},
%plural={\textit{#4}},
%description={\textit{#5}},
%first={\textit{#5}(nachfolgend #3)\glsadd{_#1}},
%firstplural={\textit{#6}(nachfolgend \glsentryplural{#4})},
%see=[Glossar:]{_#1}
%}
%\longnewglossaryentry{_#1}{type=#2,
%name={\textit{#5}},
%plural={\textit{#6}}
%}{\hspace{0pt}\\#7}
%}


% Abkuerzungen
\newacronym{jlu}{JLU}{Justus-Liebig-Universität}
\newacronym{hrz}{HRZ}{Hochschulrechenzentrum}
\newacronym[plural=LEDs, longplural={light-emitting diodes}]{led}{LED}{light-emitting diode}
\newacronym[plural=EEPROMs, longplural={electrically erasable programmable read-only memories}]{eeprom}{EEPROM}{electrically erasable programmable read-only memory}

%  Glossareintraege
\newglossaryentry{culdesac}{name=cul-de-sac, description={passage or street closed at one end}, plural=culs-de-sac}
\newglossaryentry{elite}{name={é}lite, description={select group or class}, sort=elite}
\newglossaryentry{elitism}{name={é}litism, description={advocacy of dominance by an \gls{elite}}, sort=elitism}
\newglossaryentry{attache}{name=attaché, description={person with special diplomatic responsibilities}}

% Eintraege für Symbolliste
\newglossaryentry{ohm}{type=symbols, name={\ensuremath{\Omega}}, sort=Ohm, symbol={\ensuremath{\Omega}}, description={unit of electrical resistance}}
\newglossaryentry{angstrom}{type=symbols, name={\AA}, sort=angström, symbol={\AA}, description={non-SI unit of length}}




%%%%%%%%%%%%%%%% TRIPOD %%%%%%%%%%%%%%%%
\newglossaryentry{tripod}{name=[Trip]od, description={Name der zu entwickelnden Applikation}}
\newglossaryentry{api}{name=API, description={\textit{Application Programming Interface}, Schnittstelle einer Applikation zur Nutzung über andere Programme}}
\fi

\logofilename{images/logos/SoE/de/de-soe-cmyk.png}
\projecttype{PA}
\major{HS16 Studiengang Informatik}
\title{Vacationplanner}
\shorttitle{Vacationplanner}
\author{Gruppe 15}
\authors{Gruppe 15: Fabio Costi, Raphael Emberger,\\Nicolas Kaiser, Julian Visser}
\mainreferee{Daniel Liebhart}
\referee{Liby Kunthrayil}
\industrypartner{}
\extreferee{}
\setdate{\today}

\newcommand{\AddRequirement}[2]{
\textbf{/#1#2/}
}
\newcommand{\F}[1]{
\AddRequirement{F1.}{#1}
}
\newcommand{\W}[1]{
\AddRequirement{F2.}{#1}
}
\newcommand{\A}[1]{
\AddRequirement{F3.}{#1}
}
\newcommand{\TF}[1]{
\AddRequirement{T.}{#1}
}

\begin{document}

\maketitle

\tableofcontents

\chapter{Einleitung}\label{chp:Introduction}
Geplant ist ein Ferienplaner namens "[Trip]od".\\
Der Ferienplaner dient zur Verwaltung und effizienten Planung von Ferien für Einzelpersonen oder ganzen Gruppen. Die Applikation dient also zur automatischen Datensammlung relevanter Daten um möglichst kostengünstig die Ferien geniessen zu können. Dazu wird es eine Webseite als Schnittstellenplattform geben, welche die gesamte Verwaltung übernimmt. Auf dem Server läuft dann die gesamte Logik wie API-Aufrufe, Datenverwaltung, Filterung oder Ausgabe.
\chapter{Zielbestimmungen}\label{chp:DefinitionOfGoals}
\begin{table}
\begin{center}
\begin{tabular}{l|l|l}
Kriterium & Kurzbeschrieb & Beschreibung\\\hline
\F{0010} & Account anlegen & E-Mail, Passwort
\end{tabular}
\end{center}
\end{table}
\section{Musskriterien}\label{sec:MustCriteria}
\section{Wunschkriterien}\label{sec:WishCriteria}
\section{Abgrenzungskriterien}\label{sec:DistinctionCriteria}
\chapter{Produkteinsatz}\label{chp:ProductApplication}
\section{Anwendungsbereiche}\label{sec:FieldOfApplience}
\section{Benutzergruppen des Produktes}\label{sec:TargetAudience}
\chapter{Funktionale Anforderungen}\label{chp:FunctionalRequirements}
\chapter{Funktionsbaum}\label{chp:FunctionTree}
\chapter{Nicht-Funktionale Anforderungen}\label{chp:NonFunctionalRequirements}
\chapter{Abnahmekriterien Anforderungen}\label{chp:TestRequirements}
\chapter{Verzeichnisse}\label{chp:Index}
\printglossary\label{sec:Glossar}
\newpage
\bibliography{reference}\label{sec:Bibliography}
\end{document}