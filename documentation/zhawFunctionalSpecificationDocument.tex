%%	build-queue:
%%	
%%	¦¦¦ very first run. No .ist files yet
%%	¦¦	index, citation/bibliography or glossary changed
%%	¦	every change apart from the above mentioned. Double run for labels and toc.
%%	
%%	¦¦¦xelatex
%%	¦¦makeglossaries
%%	¦¦makeindex
%%	¦¦bibtex
%%	¦xelatex
%%	¦xelatex
%%	

\RequirePackage[l2tabu,orthodox]{nag}
\documentclass[10pt,a4paper,titlepage,twoside,german,final]{zhawreprt}

\include{packages}
\if false
%\newcommand{\FullGlossaryEntry}[7]{%Abkz., Typ, Name, Name(plural), Beschr., Beschr.(plural), Erkl.
%\newglossaryentry{#1}{type=\acronymtype,
%name={\textit{#3}},
%plural={\textit{#4}},
%description={\textit{#5}},
%first={\textit{#5}(nachfolgend #3)\glsadd{_#1}},
%firstplural={\textit{#6}(nachfolgend \glsentryplural{#4})},
%see=[Glossar:]{_#1}
%}
%\longnewglossaryentry{_#1}{type=#2,
%name={\textit{#5}},
%plural={\textit{#6}}
%}{\hspace{0pt}\\#7}
%}


% Abkuerzungen
\newacronym{jlu}{JLU}{Justus-Liebig-Universität}
\newacronym{hrz}{HRZ}{Hochschulrechenzentrum}
\newacronym[plural=LEDs, longplural={light-emitting diodes}]{led}{LED}{light-emitting diode}
\newacronym[plural=EEPROMs, longplural={electrically erasable programmable read-only memories}]{eeprom}{EEPROM}{electrically erasable programmable read-only memory}

%  Glossareintraege
\newglossaryentry{culdesac}{name=cul-de-sac, description={passage or street closed at one end}, plural=culs-de-sac}
\newglossaryentry{elite}{name={é}lite, description={select group or class}, sort=elite}
\newglossaryentry{elitism}{name={é}litism, description={advocacy of dominance by an \gls{elite}}, sort=elitism}
\newglossaryentry{attache}{name=attaché, description={person with special diplomatic responsibilities}}

% Eintraege für Symbolliste
\newglossaryentry{ohm}{type=symbols, name={\ensuremath{\Omega}}, sort=Ohm, symbol={\ensuremath{\Omega}}, description={unit of electrical resistance}}
\newglossaryentry{angstrom}{type=symbols, name={\AA}, sort=angström, symbol={\AA}, description={non-SI unit of length}}




%%%%%%%%%%%%%%%% TRIPOD %%%%%%%%%%%%%%%%
\newglossaryentry{tripod}{name=[Trip]od, description={Name der zu entwickelnden Applikation}}
\newglossaryentry{api}{name=API, description={\textit{Application Programming Interface}, Schnittstelle einer Applikation zur Nutzung über andere Programme}}
\fi

\logofilename{images/logos/SoE/de/de-soe-cmyk.png}
\projecttype{PA}
\major{HS16 Studiengang Informatik}
\title{Vacationplanner}
\shorttitle{Vacationplanner}
\author{Gruppe 15}
\authors{Gruppe 15: Fabio Costi, Raphael Emberger,\\Nicolas Kaiser, Julian Visser}
\mainreferee{Daniel Liebhart}
\referee{Liby Kunthrayil}
\industrypartner{}
\extreferee{}
\setdate{\today}

\newcommand{\AddRequirement}[2]{
\textbf{/#1#2/}
}
\newcommand{\F}[1]{
\AddRequirement{F1.}{#1}
}
\newcommand{\W}[1]{
\AddRequirement{F2.}{#1}
}
\newcommand{\D}[1]{
\AddRequirement{F3.}{#1}
}
\newcommand{\R}[1]{
\AddRequirement{F4.}{#1}
}
\newcommand{\TF}[1]{
\AddRequirement{T.}{#1}
}
\newcommand{\tableheader}[2]{\multicolumn{#1}{c}{\raisebox{-0.3em}[0ex][0ex]{\large{\textbf{#2}}}}}

\numberwithin{table}{chapter}
%\renewcommand{\thetable}{\arabic{chapter}.\arabic{table}}

\begin{document}

\makezhawtitle{
\begin{center}
  \includegraphics[width=0.8\linewidth]{images/IconandText.png}
  \label{fig:TripPodLogo}
\end{center}
}

\tableofcontents

\chapter{Einleitung}\label{chp:Introduction}
Unsere Gruppe hat beim Brainstorming festgestellt, dass es noch kein geeignetes Werkzeug gibt, mit welchem man seine Ferien planen kann, sämtliche Informationen an einem Ort gesammelt sind und auf diese Informationen von jedem Mitglied der Reisegruppe jederzeit zugegriffen werden kann. Aus diesem Grund haben wir beschlossen, diese Lücke zu schliessen.\\
Geplant ist in unserem Projekt eine Webapplikation, mit welcher der Nutzer bequem seine künftigen Ferien planen kann.\\
Der Ferienplaner trägt den Namen \gls{tripod}, welcher sich aus folgenden drei Punkten zusammensetzt:
\begin{itemize}
\item TRIP: steht für die kommende Reise
\item POD: englisches Sprichwort "like two peas in a pod" (übersetzt: wie zwei Erbsen in einer Schote), steht in unserem Fall dafür, dass die Reisenden alle eine gemeinsame Faszination teilen
\item TRIPOD: das Stativ, auf welchem das Fernrohr befestigt ist, um auf der Reise in die Ferne zu schauen
\end{itemize}
Der Nutzer kann sich auf der Webseite registrieren und sich einloggen. Anschliessend kann die Planung gestartet werden. Man legt zuerst die Ferien an und kann, wenn dies gewünscht ist, beliebig Freunde dazu einladen. Diese Freunde müssen sich auch registrieren und können dann bei der Planung mitwirken. Im nächsten Schritt einigt man sich auf einen Zeitraum, in dem die Ferien stattfinden, auf ein Reiseziel und ein Budget.\\
Nun informiert sich jeder Teilnehmer über mögliche Verkehrsmittel, um zu der gewünschten Destination zu gelangen und über mögliche Unterkünfte. Diese Suche findet ausserhalb unserer Webseite statt. Ein Wunschziel von uns ist es, dass wir über verschiedene \gls{api}'s der Anbieter (Schnittstellen zu deren Webseiten) die Suche direkt auf unserer Webseite ermöglichen können.\\
Wenn ein Nutzer ein attraktives Angebot für ein Verkehrsmittel oder eine Unterkunft gefunden hat, kann er dieses auf unserer Seite eintragen. Dieses Angebot ist dann für jeden Teilnehmer einsehbar. Diese Angebote können dann in der Gruppe mittels der Kommentar-Funktion besprochen werden. Für das definitive Buchen der Reise bietet unsere Webseite keine Funktion an. Dies geschieht weiterhin über die jeweiligen Anbieter.\\
Da während der Reise oft kein Internet verfügbar ist, bieten wir eine Funktion an, mit welcher sich alle Informationen exportieren lassen und sie somit jederzeit auch offline zur Verfügung stehen.

\newpage
\section{Visualisierungsideen}\label{sec:Visualisierungsideen}
Zur Visualisierung der Applikation haben wir noch ein Wireframe, sowie einige Sketches verschiedener Applikationsseiten erstellt. Diese dienen  als ersten Entwurf. Die endgültige Version kann davon abweichen.
\begin{figure}[ht!]
  \includegraphics[width=\linewidth]{images/wireframe.png}
  \caption{Wireframe Übersicht}
  \label{fig:TripPodLogo2}
\end{figure}
\begin{figure}[ht!]
  \includegraphics[width=\linewidth]{images/wire_Welcome.png}
  \caption{Sketch Willkommensseite}
  \label{fig:TripPodLogo3}
\end{figure}
\begin{figure}[ht!]
  \includegraphics[width=\linewidth]{images/wire_Trip.png}
  \caption{Sketch Ferienplanungsseite}
  \label{fig:TripPodLogo4}
\end{figure}


\chapter{Zielbestimmungen}\label{chp:DefinitionOfGoals}
\section{Musskriterien}\label{sec:MustCriteria}
Die Applikation muss Accounts verwalten können. Dafür ist das Anlegen des Accounts und das An- und Abmelden unabdingbar. Darüber hinaus muss es möglich sein, den eigenen Account zu bearbeiten: Passwort, Name und weitere persönliche Daten soll der Benutzer ändern können. Falls das Passwort vergessen wird, soll eine Funktion zur Passwort-Rücksetzung angeboten werden.\\
Ein Account muss Ferien anlegen können und weitere Personen zu diesen Ferien hinzufügen können. Es müssen Reisedaten bearbeitbar sein. Weiter soll man ein Budget festlegen, Destinationen hinzufügen, bearbeiten und entfernen können. Selbiges muss für Verkehrsmittel und Unterkünfte gelten. Alle Einträge müssen kommentierbar sein und alle relevanten Daten müssen exportierbar sein.
\begin{table}[ht]\centering
\begin{tabular}{l}\hline
\tableheader{1}{Account verwalten}\\[0.3em]\hline
\textbf{Kriterium}\\\hline
Account anlegen\\\hline
Account bearbeiten (Passwort ändern, persönliche Daten ausfüllen)\\\hline
Benutzer an- und abmelden\\\hline
Passwort wiederherstellen\\\hline
\end{tabular}
\caption{Musskriterien Account}\label{tbl:MustAccount}
\end{table}

\begin{table}[ht]\centering
\begin{tabular}{l}\hline
\tableheader{1}{Ferien verwalten}\\[0.3em]\hline
\textbf{Kriterium}\\\hline
Ferien anlegen\\\hline
Personen hinzufügen und entfernen\\\hline
Reisedaten festlegen\\\hline
Budget festlegen\\\hline
Destination hinzufügen, bearbeiten, entfernen \\\hline
Verkehrsmittel hinzufügen, bearbeiten, entfernen\\\hline
Unterkunft hinzufügen, bearbeiten, entfernen\\\hline
Kommentar-Funktion\\\hline
Offline-Export\\\hline
\end{tabular}
\caption{Musskriterien Ferien}\label{tbl:MustVacation}
\end{table}\newpage
\section{Wunschkriterien}\label{sec:WishCriteria}
Um den Funktionsumfang bei verbleibender Zeit zu erweitern, wurden entsprechende Wunschkriterien definiert.\\
Es wäre wünschenswert, Dokumente anhängen zu können(Bsp. Booking-Bestätigungen o.Ä.). Es soll auch das Hinzufügen von mehreren Destinationen möglich sein. Die Abfrage von Flugdaten oder Unterkünften automatisiert anzubieten wäre ebenfalls erstrebsam.
\begin{table}[ht]\centering
\begin{tabular}{l}\hline
\tableheader{1}{Ferien verwalten}\\[0.3em]\hline
\textbf{Kriterium}\\\hline
Dokumente anhängen\\\hline
Mehrere Destinationen\\\hline
API-basierte Flugsuche\\\hline
API-basierte Unterkunftssuche\\\hline
\end{tabular}
\caption{Wunschkriterien Ferien}\label{tbl:WishVacation}
\end{table}
\section{Abgrenzungskriterien}\label{sec:DistinctionCriteria}
Die in diesem Kapitel definierten Abgrenzungskriterien sollen bewusst nicht erreicht werden.\\
Das tatsächliche Booking von Käufen oder Buchungen sind nicht zu unterstützen, da dies Verantwortungen auflädt, welche nicht getragen werden sollen. Weiter sollen keine Verknüpfungen zu Social-Media-Plattformen(Facebook, Twitter, Tumbler etc.) eingebaut werden. Auch auf Werbung wird in diesem Stadium aus Gründen des Konsumenten-Komforts gänzlich verzichtet.
\begin{table}[ht]\centering
\begin{tabular}{l|p{2.5cm}|p{8cm}}\hline
\tableheader{3}{Allgemein}\\[0.3em]\hline
\textbf{Kriterium} & \textbf{Kurzbeschrieb} & \textbf{Beschreibung}\\\hline
\D{1000} & Booking & Die tatsächliche Tätigung von Käufen oder Buchungen soll nicht unterstützt werden. Die Buchung geschieht über Drittanbieter.\\\hline
\D{1010} & Social-Media & Verknüpfungen auf Social-Media-Plattformen werden nicht unterstützt. Ebenso ist kein Login über einen Social-Media-Account möglich.\\\hline
\D{1020} & Werbung & Es wird keine Werbung eingebaut. Der komplette Funktionsumfang der Webseite ist kostenlos verfügbar.\\\hline
\D{1030} & Mehrsprachigkeit & Die Webseite ist lediglich in der deutschen Sprache verfügbar. Die Implementation einer zusätzlichen Sprache ist für den ersten Release nicht geplant.\\\hline
\end{tabular}
\caption{Abgrenzungskriterien Allgemein}\label{tbl:DistinctionGeneral}
\end{table}
\chapter{Produkteinsatz}\label{chp:ProductApplication}
\section{Anwendungsbereiche}\label{sec:FieldOfApplience}
Genutzt wird der Ferienplaner über das Web. Dies ermöglicht, dass mehrere Personen problemlos zusammen ihre Reise planen können.\\
Theoretisch kann jeder Mensch, der Zugang zum Internet hat, unsere Applikation benutzen, jedoch ist vorerst nur eine deutsche Webseite geplant, sodass in einem ersten Schritt auf deutschsprachige Nutzer gezielt wird.
\section{Benutzergruppen des Produktes}\label{sec:TargetAudience}
Das Produkt gilt der öffentlichen, nicht-kommerziellen Nutzung und ist für Gruppen wie auch Einzelpersonen geeignet. Dies umfasst Personen jeden Geschlechtes oder Alters. Ein wichtiger Punkt unserer Webseite ist auch das Budget, da gerade wir Studenten oftmals nicht über grosse, finanzielle Möglichkeiten verfügen. Somit ist unsere Applikation bestens geeignet für reiselustige Personen, welche gerne ihre Reise selber planen oder sich den Luxus eines Reisebüros nicht leisten können.

\chapter{Funktionale Anforderungen}\label{chp:FunctionalRequirements}
Der Ferienplaner ist über einen eigenen Account zu verwalten. Jeder Reisende braucht einen eigenen Account (Dummy-Accounts für Gruppen könnten eine Alternative sein). Mit diesem Account lassen sich neue Ferien anlegen, denen andere Accounts hinzugefügt werden können. Bei Gruppen ist der Gruppenersteller automatisch der Gruppenadmin. Ferien können nun terminlich eingeschränkt, Abreise- und Rückkehrdaten festgelegt und Reiseziele bestimmt werden. Weitere Einstellmöglichkeiten (Minimum-Budget, Maximum-Komfort, etc.) können ebenfalls bestimmt werden. All diese Parameter können vom Admin uneingeschränkt und von Gruppenmitgliedern nach Revision des Admins und anderer Gruppenmitglieder aufgenommen werden.\\
Nun kann auf Knopfdruck die Berechnung gestartet werden. Dabei werden über die API's von mehreren Anbietern Angebote gesammelt und Rechnungen getätigt. Der Output soll aus den Kosten der Ideallösung (Auswahl der Möglichkeiten evtl. ebenfalls möglich), einer Tabelle der buchbaren Mittel und einem Zeitplan bestehen. Das effektive Buchen geht dann aus Sicherheitsgründen weiterhin händisch über den Admin.

\begin{table}[ht]\centering
\begin{longtable}{l|p{2.5cm}|p{5cm}|p{2cm}|p{1.5cm}}\hline
\tableheader{5}{Account verwalten}\\[0.3em]\hline
\textbf{Kriterium} & \textbf{Kurzbeschrieb} & \textbf{Beschreibung} & \textbf{Komplexität} & \textbf{Priorität}\\\hline
\F{1000} & Account anlegen & Es kann ein Account mit den folgenden Daten angelegt werden:\linebreak
\begin{itemize}
\item E-Mail Adresse [zwingend]
\item Passwort [zwingend]
\item Name (Vor- und Nachname) [zwingend]
\item Telefonnummer [fakultativ]
\end{itemize}
& mittel & hoch\\\hline
Akzeptanzkriterium \F{1000}:\linebreak
Ein Account kann nur erstellt werden, wenn die noch kein anderer Account mit der angegebenen E-Mail Adresse existiert. Die E-Mail Adresse muss dem E-Mail Format ensprechen und das Passwort muss eine gewisse Sicherheit bieten (Buchstaben, Ziffern sowie Sonderzeichen).\\\hline
\F{1010} & Account bearbeiten & Das Passwort und die persönlichen Daten können angepasst werden. Einzig die E-Mail Adresse kann nicht verändert werden. & tief & hoch\\\hline
Akzeptanzkriterium \F{1010}:\linebreak
Es können alle Daten bis auf die E-Mail Adresse nach Belieben angepasst werden. Die Änderungen werden sofort übernommen und sind mit dem Abspeichern aktiv. Das Passwort muss weiterhin den Sicherheitsrichtlinien entsprechen.\\\hline
\F{1020} & Benutzer anmelden & Ein Benutzer kann sich mit seinem Account (E-Mail und Passwort) anmelden. & mittel & hoch\\\hline
Akzeptanzkriterium \F{1020}:\linebreak
Die Anmeldung erfolgt über die Eingabe der korrekten Kombination aus E-Mail Adresse und Passwort.\\\hline
\F{1030} & Benutzer abmelden & Ein Benutzer kann sich, wenn er angemeldet ist, auch wieder abmelden. & tief & hoch\\\hline
Akzeptanzkriterium \F{1030}:\linebreak
Damit sich ein Benutzer abmelden kann, muss er sich zuvor angemeldet haben und immer noch angemeldet sein.\\\hline
\F{1040} & Passwort wiederherstellen & Wenn ein Benutzer sein Passwort vergessen hat, kann er mittels Eingabe seiner E-Mail Adresse ein neues Passwort anfordern. & tief & hoch\\\hline
Akzeptanzkriterium \F{1040}:\linebreak
Wenn eine E-Mail Adresse eingegeben wird, welche einem Account zugeordnet werden kann, wird das Passwort zurückgesetzt und ein neues Passwort wird an die angegebene E-Mail Adresse versandt.\\\hline
\end{longtable}
\caption{Funktionale Anforderungen Account}\label{tbl:FuncAccount}
\end{table}

\begin{center}
\begin{longtable}{l|p{2.5cm}|p{5cm}|p{2cm}|p{1.5cm}}\hline
\tableheader{5}{Ferien verwalten}\\[0.3em]\hline
\textbf{Kriterium} & \textbf{Kurzbeschrieb} & \textbf{Beschreibung} & \textbf{Komplexität} & \textbf{Priorität}\\\hline
\F{2000} & Ferien anlegen & Jeder Benutzer kann Ferien erstellen. Eine Ferienplanung beinhaltet folgende Informationen:\linebreak
\begin{itemize}
\item Titel [zwingend]
\item Beschreibung [zwingend]
\item Ersteller [zwingend]
\item Personen [fakultativ, kann auch nur der "Ersteller" sein]
\item Reisedaten (An- und Abreisedatum) [zwingend]
\item Budget [fakultativ]
\item Destinationen [mindestens Eine]
\item Verkehrsmittel [fakultativ]
\item Unterkünfte [fakultativ]
\item Kommentare [fakultativ]
\end{itemize}
& tief & hoch\\\hline
Akzeptanzkriterium \F{2000}:\linebreak
Die Ferienplanung kann dann erstellt werden, wenn die gemäss in der Beschreibung beschriebenen, zwingenden Felder ausgefüllt wurden.\\\hline
\F{2010} & Personen hinzufügen & Jeder Benutzer kann zusätzliche Personen (registrierte Benutzer) per E-Mail Adresse zur Ferienplanung hinzufügen. Die hinzugefügte Person wird mittels E-Mail über die neu verfügbare Ferienplanug informiert. & tief & hoch\\\hline
Akzeptanzkriterium \F{2010}:\linebreak
Eine Person kann zur Ferienplanung hinzugefügt werden, wenn ein Account mit der angegebenen E-Mail Adresse existiert. Die hinzugefügte Person soll mittels einer E-Mail darüber informiert werden.\\\hline
\F{2020} & Personen entfernen & Jeder Benutzer kann sich oder andere Personen (registrierte Benutzer) entfernen. Die entfernte Person wird mittels E-Mail über den Ausschluss aus der Ferienplanung informiert. & tief & hoch\\\hline
Akzeptanzkriterium \F{2020}:\linebreak
Eine Person kann von der Ferienplanung entfernt werden, wenn sie einmal hinzugefügt wurde. Jeder Benutzer kann jede Person (inklusive sich selbst) aus der Ferienplanung entfernen.\\\hline
\F{2030} & Reisedaten festlegen & Der Zeitraum der Ferien kann festgelegt und verändert werden. & tief & hoch\\\hline
Akzeptanzkriterium \F{2030}:\linebreak
Die Reisedaten sollen mittels einem Date-Picker ausgewählt werden und im entsprechenden Format auch in der Datenbank abgelegt werden. Es kann kein Datum von Hand eingegeben werden, um die Konsistenz der Daten zu gewährleisten.\\\hline
\F{2040} & Budget festlegen & Das Budget kann festgelegt und verändert werden. Dieses besteht aus:
\begin{itemize}
\item Budget für die Reise
\item Budget für die Unterkunft
\end{itemize}
& tief & hoch\\\hline
Akzeptanzkriterium \F{2040}:\linebreak
Das Budget für die Reise sowie für die Unterkunft soll in einem Feld definiert werden, welches nur Zahlen akzeptiert. Eine Währung soll aktuell nicht spezifiert werden, sie ist standardmässig CHF.\\\hline
\F{2050} & Destination hinzufügen & Jeder Benutzer kann eine Destination hinzufügen. & tief & hoch\\\hline
Akzeptanzkriterium \F{2050}:\linebreak
Eine Destination kann von Hand hinzugefügt werden. Sie kann aus einer vorgegebenen Liste ausgewählt oder manuell eingetragen werden. Die Eingabe darf nicht leer sein.\\\hline
\F{2060} & Destination bearbeiten & Jeder Benutzer kann eine Destination bearbeiten. & tief & hoch\\\hline
Akzeptanzkriterium \F{2060}:\linebreak
Eine Destination kann durch das Auswählen eines anderen Eintrages aus der Vorgabeliste oder durch eine manuelle Eingabe bearbeitet werden.\\\hline
\F{2070} & Destination entfernen & Jeder Benutzer kann eine Destination entfernen. & tief & hoch\\\hline
Akzeptanzkriterium \F{2070}:\linebreak
Eine Destination kann entfernt werden, insofern es nicht die einzige definiert Destination in der Ferienplanung ist. Es muss immer mindestens ein Destinationseintrag pro Ferienplanung existieren.\\\hline
\F{2080} & Verkehrsmittel hinzufügen & Jeder Benutzer kann ein Verkehrsmittel hinzufügen. Hierzu kann ein Link des jeweiligen Angebotes hinterlegt werden. & tief & hoch\\\hline
Akzeptanzkriterium \F{2080}:\linebreak
Jeder Benutzer einer Ferienplanung kann ein neues Verkehrsmittel hinzufügen. Dabei muss die Art des Verkehrsmittel aus einer vorgegebenen Liste ausgewählt werden, zusätzlich kann noch der Link eines dazugehörigen Angebotes oder eine Beschreibung hinterlegt werden. Es können auch mehrere Verkehrsmittel derselben Art in einer Ferienplanung erstellt werden.\\\hline
\F{2090} & Verkehrsmittel bearbeiten & Jeder Benutzer kann ein Verkehrsmittel bearbeiten. & tief & hoch\\\hline
Akzeptanzkriterium \F{2090}:\linebreak
Die erfassten Verkehrsmittel können bearbeitet werden, indem eine andere Art des Verkehrsmittel ausgewählt wurde, oder der Link respektive die Beschreibung angepasst wurde.\\\hline
\F{2100} & Verkehrsmittel entfernen & Jeder Benutzer kann ein Verkehrsmittel entfernen. & tief & hoch\\\hline
Akzeptanzkriterium \F{2100}:\linebreak
Ein Verkehrsmittel kann jederzeit entfernt werden. Es gibt keine Mindestanzahl an Verkehrsmitteln in einer Ferienplanung.\\\hline
\F{2110} & Unterkunft hinzufügen & Jeder Benutzer kann eine Unterkunft hinzufügen. Hierzu kann ein Link des jeweiligen Angebotes hinterlegt werden. & tief & hoch\\\hline
Akzeptanzkriterium \F{2110}:\linebreak
Jeder Benutzer einer Ferienplanung kann eine neue Unterkunft hinzufügen. Dabei muss die Art der Unterkunft aus einer vorgegebenen Liste ausgewählt werden, zusätzlich kann noch der Link eines dazugehörigen Angebotes oder eine Beschreibung hinterlegt werden. Es können auch mehrere Unterkünfte derselben Art in einer Ferienplanung erstellt werden.\\\hline
\F{2120} & Unterkunft bearbeiten & Jeder Benutzer kann eine Unterkunft bearbeiten. & tief & hoch\\\hline
Akzeptanzkriterium \F{2120}:\linebreak
Die erfassten Unterkünfte können bearbeitet werden, indem eine andere Art der Unterkunft ausgewählt wurde, oder der Link respektive die Beschreibung angepasst wurde.\\\hline
\F{2130} & Unterkunft entfernen & Jeder Benutzer kann eine Unterkunft entfernen. & tief & hoch\\\hline
Akzeptanzkriterium \F{2130}:\linebreak
Eine Unterkunft kann jederzeit entfernt werden. Es gibt keine Mindestanzahl an Unterkünften in einer Ferienplanung.\\\hline
\F{2140} & Kommentar-Funktion & Jeder Benutzer kann Kommentare hinterlassen (Chat). & mittel & hoch\\\hline
Akzeptanzkriterium \F{2140}:\linebreak
Die Eingabe eines Kommentares, erfolgt über ein Textfeld und sie darf nicht leer sein.\\\hline
\F{2150} & Offline-Export & Jeder Benutzer kann einen Export mit den wichtigsten Daten der Ferien herunterladen. & mittel & hoch\\\hline
Akzeptanzkriterium \F{2150}:\linebreak
Alle relevanten Daten der Ferienplanung sollen in einem angemessenen und klar strukturierten Export (z.B. csv) heruntergeladen werden können.\\\hline
\W{2000} & Dokumente anhängen & Booking-Bestätigungen und ähnliche Dokumente sollen abgelegt werden können. & hoch & tief\\\hline
Akzeptanzkriterium \W{2000}:\linebreak
Es sollen Dokumente, egal welches Datenformates in einer Ferienplanung abgelegt werden können. Diese sollen für alle Benutzer einer Ferienplanung verfügbar sein.\\\hline
\W{2010} & Mehrere Destinationen & Es soll möglich sein, eine Planung für mehrere Destinationen hintereinander zu machen. & hoch & tief\\\hline
Akzeptanzkriterium \W{2010}:\linebreak
Wenn mehrere Destinationen aneinander gehängt werden, darf jeder Destinationseintrag nur einmal verwendet werden. Das heisst man muss für jeden Zwischenstopp eine eigene Destination gemäss Vorgaben erstellen.\\\hline
\W{2020} & API-basierte Flugsuche & Es soll möglich sein, Flugangebote über eine API abzurufen. z.B. Google Flights & hoch & tief\\\hline
Akzeptanzkriterium \W{2020}:\linebreak
Durch einen Button-Klick soll die automatisierte Flugsuche gestartet werden. Die verschiedenen Parameter dazu werden aus der Ferienplanung ausgelesen. Danach werden die Einträge in eine separate Liste abgefüllt. Es wird die Anzahl Einträge angezeigt, sowie jeder Eintrag ist ein Link zu einem passenden Angebot.\\\hline
\W{2030} & API-basierte Unterkunftssuche & Es soll möglich sein, Unterkünfte über eine API abzurufen. z.B. Booking.com & hoch & tief\\\hline
Akzeptanzkriterium \W{2030}:\linebreak
Durch einen Button-Klick soll die automatisierte Unterkunftssuche gestartet werden. Die verschiedenen Parameter dazu werden aus der Ferienplanung ausgelesen. Danach werden die Einträge in eine separate Liste abgefüllt. Es wird die Anzahl Einträge angezeigt, sowie jeder Eintrag ist ein Link zu einem passenden Angebot.\\\hline
\caption{Funktionale Anforderungen Ferien}\label{tbl:FuncVacation}
\end{longtable}
\end{center}\newpage

\chapter{Funktionsbaum}\label{chp:FunctionTree}
\begin{minipage}{\textwidth}
%\begin{landscape}
\definecolor{MainBack1}{RGB}{66,66,68}
\definecolor{MainBack2}{RGB}{86,86,88}
\definecolor{MainFront}{RGB}{220,220,220}
\definecolor{SubBack1}{RGB}{240,240,240}
\definecolor{SubBack2}{RGB}{255,255,255}
\definecolor{SubFront}{RGB}{46,46,48}
\newcommand{\NewMainNode}[3]{ %NewNodeName, Pos, Text
\node(#1)[MainNode] at (#2) {#3};
}
\newcommand{\AddMainNode}[5]{ %NewNodeName, Parent, XOff, YOff, Text
\node(#1)[MainNode,below=of #2,shift={(#3, #4)}] {#5};
\coordinate(Temp) at ($(#1)-(#3, -1.3cm)$);
\draw[line](#2) -- (Temp) -| (#1);
}
\newcommand{\CreateNodeOffset}[2]{ %NewNodeName, Parent
\path (#2) + (30pt, -20.5pt) coordinate (#1);
}
\newcommand{\NewSubNode}[3]{ %NewNodeName, Parent, Text
\node(#1)[SubNode,below=of #2,shift={(40pt, 0pt)}] {#3};
\draw[line](#2) ++ (SubOffset) |- (#1);
}
\newcommand{\AddSubNode}[4]{ %NewNodeName, Parent, BelowSubNode Text
\node(#1)[SubNode,below=of #3] {#4};
\draw[line](#2) ++ (SubOffset) |- (#1);
}
\begin{turn}{90}
\begin{tikzpicture}[
text height = 0.1cm,
every node/.style = {
	column sep = 0.5cm,
	row sep = 1cm,
	thick,
	node distance = 1em,
	align = center,
	font = \small
},
base/.style = {
	text centered,
	inner sep = 1pt,
	minimum height = 30pt,
	minimum width = 100pt,
},
MainNode/.style = {
	base,
	rectangle,
	bottom color = MainBack1,
	top color = MainBack2,
	text = MainFront,
	draw = MainBack2
},
SubNode/.style = {
	base,
	rectangle,
	bottom color = SubBack1,
	top color = SubBack2,
	text = SubFront,
	draw = SubFront
},
line/.style = {
	draw,
	-stealth,
	shorten > = 2pt,
	thick
}
]

\coordinate(SubOffset) at (-30pt,-15pt);
\coordinate(MainOffset) at (12cm, 0cm);

%% LAYOUT
\NewMainNode{F2}{0, 0}{Ferien verwalten}

\NewSubNode{F2000}{F2}{Ferien anlegen}
\AddSubNode{F20100}{F2}{F2000}{Personen verwalten}

\node(F2010)[SubNode,right=of F20100,shift={(1cm, 0pt)}] {Personen hinzufügen};
\draw[line](F20100) |- (F2010);

\node(F2020)[SubNode,below=of F2010] {Personen abmelden};
\draw[line](F20100) ++ (2.5cm, 0pt) |- (F2020);

\AddSubNode{F2020}{F2}{F20100}{Daten festlegen}
\AddSubNode{F2030}{F2}{F2020}{Budget festlegen}
\AddSubNode{F20400}{F2}{F2030}{Destinationen verwalten}

\node(F2040)[SubNode,right=of F20400,shift={(11cm, 0pt)}] {Destinationen hinzufügen};
\draw[line](F20400) |- (F2040);
\node(F2050)[SubNode,below=of F2040] {Destinationen bearbeiten};
\draw[line](F20400) ++ (12.5cm, 0pt) |- (F2050);
\node(F2060)[SubNode,below=of F2050] {Destinationen entfernen};
\draw[line](F20400) ++ (12.5cm, 0pt) |- (F2060);

\AddSubNode{F20700}{F2}{F20400}{Verkehrsmittel verwalten}

\node(F2070)[SubNode,right=of F20700,shift={(6cm, 0pt)}] {Verkehrsmittel hinzufügen};
\draw[line](F20700) |- (F2070);
\node(F2080)[SubNode,below=of F2070] {Verkehrsmittel bearbeiten};
\draw[line](F20700) ++ (7.5cm, 0pt) |- (F2080);
\node(F2090)[SubNode,below=of F2080] {Verkehrsmittel entfernen};
\draw[line](F20700) ++ (7.5cm, 0pt) |- (F2090);

\AddSubNode{F21000}{F2}{F20700}{Unterkunft verwalten}

\node(F2100)[SubNode,right=of F21000,shift={(1cm, 0pt)}] {Unterkunft hinzufügen};
\draw[line](F21000) |- (F2100);
\node(F2110)[SubNode,below=of F2100] {Unterkunft bearbeiten};
\draw[line](F21000) ++ (2.5cm, 0pt) |- (F2110);
\node(F2120)[SubNode,below=of F2110] {Unterkunft entfernen};
\draw[line](F21000) ++ (2.5cm, 0pt) |- (F2120);

\AddSubNode{F2130}{F2}{F21000}{Kommentar-Funktion}
\AddSubNode{F2140}{F2}{F2130}{Dokumente anhängen}


\NewMainNode{F1}{$(F2)+(MainOffset)$}{Account}
\NewSubNode{F1010}{F1}{Account anlegen}
\AddSubNode{F1020}{F1}{F1010}{Account bearbeiten}

\end{tikzpicture}
\end{turn}
%\end{landscape}
\end{minipage}
\chapter{Nicht-Funktionale Anforderungen}\label{chp:NonFunctionalRequirements}
\begin{table}[ht]\centering
\begin{tabular}{l|p{3.0cm}|p{5cm}|p{2cm}|p{1.25cm}}\hline
\textbf{Kriterium} & \textbf{Kurzbeschrieb} & \textbf{Beschreibung} &\textbf{Komplexität} & \textbf{Priorität}\\\hline
\R{1000} & Gebrauchsfähigkeit (Usability) & Die Webseiten müssen durch den Benutzer, welcher dem Benutzerprofil entspricht, ohne weitere Hilfe verwendet werden können. & mittel & hoch\\
\R{1010} & Fehlertoleranz & Hinweise und Fehlermeldungen müssen für den Benutzer verständlich formuliert sein und eine Hilfestellung zur Problemlösung beinhalten. & mittel & mittel\\
\R{1020} & Sprache \& länderspezifische Einstellungen & Die Webseiten sind in deutscher Sprache (Schweiz) verfasst, verwenden den Zeichensatz UTF-8 und die Schweiz-spezifischen Einstellungen von Datum, Zeit, Zahlen und Währung. & tief & hoch\\
\R{1030} & Zielplattform (Server) & Die Web-Applikation muss als JavaServer-Pages auf dem zur Verfügung gestellten virtuellen Server unter Verwendung einer SQL-Datenbank mit Apache Tomcat betrieben werden. & mittel & hoch\\
\R{1040} & Zielplattform (Client) & Die Webseiten werden in der aktuellsten freigegebenen Version des Mozilla Firefox und Google Chrome korrekt dargestellt. & mittel & hoch\\
\R{1050} & Werkzeuge zur Entwicklung & Als Projektmanagement-Tool und zur Verwaltung des Sourcecode und der Dokumente muss der zur Verfügung gestellte github Server verwendet werden. & tief & hoch\\
\R{1060} & Robustheit & Auch nach einem Neustart des virtuellen Servers muss die Webseite voll funktionsfähig sein. & mittel & hoch\\
\R{1070} & Testbarkeit & Für die Durchführung der Tests und der Abnahme müssen sinnvolle Testdaten in genügendem Umfang zur Verfügung gestellt werden. & hoch & hoch\\
\R{1080} & Design & Das Design der Webseite soll modern und intuitiv sein. & mittel & hoch
\end{tabular}
\caption{Nicht-Funktionale Anforderungen}\label{tbl:NonFunctionalRequirements}
\end{table}
\chapter{Abnahmekriterien Anforderungen}\label{chp:TestRequirements}
\begin{table}[ht]\centering
\begin{tabular}{l|p{4cm}|p{8cm}}\hline
Kriterium & Kurzbeschrieb & Beschreibung\\\hline
\TF{1000} & Account anlegen und Profil befüllen & Es ist ein Account anzulegen mit dem Namen "Test Account" und der E-Mail-Addresse "account@test.me". Weiter soll das Profil wie folgt befüllt werden:\ldots
\end{tabular}
\caption{Abnahmekriterien}\label{tbl:TestRequirements}
\end{table}
\chapter{Verzeichnisse}\label{chp:Index}
\printglossaries\label{sec:Glossar}
\listoftables
\listoffigures
\printindex
\newpage
\bibliography{reference}\label{sec:Bibliography}
\end{document}