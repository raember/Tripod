%%	build-queue:
%%	
%%	¦¦¦ very first run. No .ist files yet
%%	¦¦	index, citation/bibliography or glossary changed
%%	¦	every change apart from the above mentioned. Double run for labels and toc.
%%	
%%	¦¦¦xelatex
%%	¦¦makeglossaries
%%	¦¦makeindex
%%	¦¦bibtex
%%	¦xelatex
%%	¦xelatex
%%	

\RequirePackage[l2tabu,orthodox]{nag}
\documentclass[10pt,a4paper,titlepage,twoside,german]{zhawreprt}

	\setdefaultlanguage{german}
\RequirePackage{lscape}
\usepackage{pgfgantt}
\usepackage{xcolor}
	\definecolor{lgray}{RGB}{250,250,250}
	\definecolor{lgreen}{RGB}{63,127,95}
	\definecolor{lred}{RGB}{127,0,85}
	\definecolor{lblue}{RGB}{42,0,255}
\usepackage{listings}
	\lstdefinestyle{basestyle}{
		basicstyle=\small\ttfamily,
		breakatwhitespace = true,
		tabsize = 4,
		frame = double,
		numbers = left,
		numbersep = 10pt,
		numberstyle = {\tiny\emptyaccsupp},
%		firstnumber = auto,
		numberblanklines = true,
		captionpos = b,
		columns = fullflexible,
		extendedchars = true,
		float = ht,
		showtabs = false,
		showspaces=false,
		showstringspaces=false,
		breaklines=true,
%		prebreak=\Righttorque
		backgroundcolor=\color{lgray},
		keywordstyle=\color{lred}\bfseries,
		commentstyle=\color{lgreen}\ttfamily,
%		morekeywords={printstr, printhexln},
		stringstyle=\color{lblue},
		xleftmargin = \fboxsep,
		xrightmargin = -6pt,
		showstringspaces=true,
	}
	\newcommand{\setlistingCSharp}{
		\lstset{
		style = basestyle,
		language = [Sharp]C,
		%otherkeywords = {*,<,>,=,;,\{,\}},
		%deletekeywords = {...},
	}}
	\newcommand{\setlistingCpp}{
		\lstset{
		style = basestyle,
		language = C++,
		%otherkeywords = {*,<,>,=,;,\{,\}},
		%deletekeywords = {...},
	}}
	\newcommand{\setlistingJava}{
		\lstset{
		style = basestyle,
		language = Java,
		%otherkeywords = {*,<,>,=,;,\{,\}},
		%deletekeywords = {...},
	}}
	\newcommand{\setlistingLaTeX}{
		\lstset{
		style = basestyle,
		language = TeX,
		%otherkeywords = {*,<,>,=,;,\{,\}},
		%deletekeywords = {...},
	}}
	\newcommand{\setlistingMatlab}{
		\lstset{
		style = basestyle,
		language = Matlab,
		otherkeywords = {methods,enumeration,properties,classdef,Sealed,Abstract},
		%deletekeywords = {...},
	}}
\usepackage{xstring}
	\newcommand{\inlist}[2]{
		\IfSubStr{,#2,}{,\arabic{#1},}{\color{lgray!95!blue}}{\color{lgray}}
	}
\usepackage{lstlinebgrd}
	\makeatletter
	\renewcommand{\lst@linebgrd}{%
	\ifx\lst@linebgrdcolor\empty\else
		\rlap{%
			\lst@basicstyle
			\color{lgray}
			\lst@linebgrdcolor{%
				\kern-\dimexpr\lst@linebgrdsep\relax%
				\lst@linebgrdcmd{\lst@linebgrdwidth}{\lst@linebgrdheight}{\lst@linebgrddepth }%
			}%
		}%
	\fi}
	\makeatother
\usepackage[space=true]{accsupp}
	\newcommand\emptyaccsupp[1]{\BeginAccSupp{ActualText={}}#1\EndAccSupp{}}
\usepackage{rotating}
\usepackage{mathptmx}
\usepackage{amssymb}
\usepackage{textcomp}
\usepackage[squaren]{SIunits}
\usepackage{amsmath}
\usepackage{amsfonts}
\usepackage{amssymb}
\usepackage[toc,page]{appendix}
\usepackage{fontspec}
	\setmainfont{Arial} % sets the roman font
	\setsansfont{Arial} % sets the sans-sérif font
	\setmonofont{Arial} % sets the monospace font
\usepackage{microtype}
\usepackage{colortbl}
\usepackage{tabularx}
\usepackage{longtable}
\usepackage{pgf,tikz}
	\usetikzlibrary{shapes}
	\usetikzlibrary{shapes.geometric}
	\usetikzlibrary{shapes.arrows}
	\usetikzlibrary{positioning}
	\usetikzlibrary{fit}
	\usetikzlibrary{calc}
	\usetikzlibrary{patterns}
\usepackage{pgfplots}
	\pgfplotsset{compat=1.13}
\usepackage{array}
\usepackage{natbib}
	\bibliographystyle{agsm}
\usepackage{usecases}
\usepackage{footnote}
	\makesavenoteenv{description}
\usepackage{makeidx}
	\makeindex
\usepackage[nottoc]{tocbibind}
\makeatletter
	\if@inltxdoc\else
	  \renewenvironment{theindex}%
	    {\if@twocolumn
	       \@restonecolfalse
	     \else
	       \@restonecoltrue
	     \fi
	     \if@bibchapter
	        \if@donumindex
	          \refstepcounter{section}
	          \twocolumn[\vspace*{2\topskip}%
	                     \@makechapterhead{\indexname}]%
	          \addcontentsline{toc}{section}{\protect\numberline{\thesection}\indexname}
	          \sectionmark{\indexname}
	        \else
	          \if@dotocind
	            \twocolumn[\vspace*{2\topskip}%
	                       \@makeschapterhead{\indexname}]%
	            \prw@mkboth{\indexname}
	            \addcontentsline{toc}{section}{\protect\numberline{\thesection}\indexname}
	          \else
	            \twocolumn[\vspace*{2\topskip}%
	                       \@makeschapterhead{\indexname}]%
	            \prw@mkboth{\indexname}
	          \fi
	        \fi
	     \else
	        \if@donumindex
	          \twocolumn[\vspace*{-1.5\topskip}%
	                     \@nameuse{\@tocextra}{\indexname}]%
	          \csname \@tocextra mark\endcsname{\indexname}
	        \else
	          \if@dotocind
	            \twocolumn[\vspace*{-1.5\topskip}%
	                       \toc@headstar{\@tocextra}{\indexname}]%
	            \prw@mkboth{\indexname}
	            \addcontentsline{toc}{\@tocextra}{\indexname}
	          \else
	            \twocolumn[\vspace*{-1.5\topskip}%
	                       \toc@headstar{\@tocextra}{\indexname}]%
	            \prw@mkboth{\indexname}
	          \fi
	        \fi
	     \fi
	   \thispagestyle{plain}\parindent\z@
	   \parskip\z@ \@plus .3\p@\relax
	   \let\item\@idxitem}
	   {\if@restonecol\onecolumn\else\clearpage\fi}
	\fi
\makeatother
\usepackage{etoolbox}
\makeatletter
	\renewcommand\listoftables{%
	    \section{Tabellenverzeichnis}%
	    \@mkboth{\MakeUppercase\listtablename}%
	        {\MakeUppercase\listtablename}%
	    \@starttoc{lot}%
	}
	\renewcommand\listoffigures{%
	    \section{Abbildungsverzeichnis}%
	    \@mkboth{\MakeUppercase\listfigurename}%
	        {\MakeUppercase\listfigurename}%
	    \@starttoc{lof}%
	}
	\renewcommand\lstlistoflistings{%
	    \section{Listingverzeichnis}%
	    \@mkboth{\MakeUppercase\listfigurename}%
	        {\MakeUppercase\listfigurename}%
	    \@starttoc{lol}%
	}
	\renewenvironment{thebibliography}[1]
	     {\section{\bibname}% <-- this line was changed from \chapter* to \section*
	      \@mkboth{\MakeUppercase\bibname}{\MakeUppercase\bibname}%
	      \list{\@biblabel{\@arabic\c@enumiv}}%
	           {\settowidth\labelwidth{\@biblabel{#1}}%
	            \leftmargin\labelwidth
	            \advance\leftmargin\labelsep
	            \@openbib@code
	            \usecounter{enumiv}%
	            \let\p@enumiv\@empty
	            \renewcommand\theenumiv{\@arabic\c@enumiv}}%
	      \sloppy
	      \clubpenalty4000
	      \@clubpenalty \clubpenalty
	      \widowpenalty4000%
	      \sfcode`\.\@m}
	     {\def\@noitemerr
	       {\@latex@warning{Empty `thebibliography' environment}}%
	      \endlist}
\makeatother
	
	\renewcommand\appendixname{Appendix}
\makeatletter
	\renewenvironment{theindex}{
		\renameindex
		\let\ps@plainorig\ps@plain
		\let\ps@plain\ps@scrheadings
		\if@twocolumn
			\@restonecolfalse
		\else
			\@restonecoltrue
		\fi
		\columnseprule \z@
		\columnsep 35\p@
		\twocolumn[\section{\indexname}]%
		\@mkboth{\MakeUppercase\indexname}{\MakeUppercase\indexname}%
		\thispagestyle{plain}\parindent\z@
		\parskip\z@ \@plus .3\p@\relax
	\let\item\@idxitem}
	{\if@restonecol\onecolumn\else\clearpage\fi}
\makeatother

\usepackage{varioref}
\usepackage[colorlinks=true, linkcolor=black, citecolor=black, plainpages=false, unicode, pdfencoding=auto ,backref=page]{hyperref}
\usepackage{cleveref}
\usepackage[automake,acronym,numberedsection]{glossaries-extra}
	\renewcommand*{\glspostdescription}{}
	\newglossary[slg]{symbols}{sym}{sbl}{Symbolverzeichnis}
	\makeglossaries
	\loadglsentries{glossaryentries.tex}
	\pdfstringdefDisableCommands{\let\textenglish\@firstofone\let\textgerman\@firstofone}
	\makeatletter
	\renewcommand*{\@@glossarysec}{section}
	\makeatother
\if false
%\newcommand{\FullGlossaryEntry}[7]{%Abkz., Typ, Name, Name(plural), Beschr., Beschr.(plural), Erkl.
%\newglossaryentry{#1}{type=\acronymtype,
%name={\textit{#3}},
%plural={\textit{#4}},
%description={\textit{#5}},
%first={\textit{#5}(nachfolgend #3)\glsadd{_#1}},
%firstplural={\textit{#6}(nachfolgend \glsentryplural{#4})},
%see=[Glossar:]{_#1}
%}
%\longnewglossaryentry{_#1}{type=#2,
%name={\textit{#5}},
%plural={\textit{#6}}
%}{\hspace{0pt}\\#7}
%}


% Abkuerzungen
\newacronym{jlu}{JLU}{Justus-Liebig-Universität}
\newacronym{hrz}{HRZ}{Hochschulrechenzentrum}
\newacronym[plural=LEDs, longplural={light-emitting diodes}]{led}{LED}{light-emitting diode}
\newacronym[plural=EEPROMs, longplural={electrically erasable programmable read-only memories}]{eeprom}{EEPROM}{electrically erasable programmable read-only memory}

%  Glossareintraege
\newglossaryentry{culdesac}{name=cul-de-sac, description={passage or street closed at one end}, plural=culs-de-sac}
\newglossaryentry{elite}{name={é}lite, description={select group or class}, sort=elite}
\newglossaryentry{elitism}{name={é}litism, description={advocacy of dominance by an \gls{elite}}, sort=elitism}
\newglossaryentry{attache}{name=attaché, description={person with special diplomatic responsibilities}}

% Eintraege für Symbolliste
\newglossaryentry{ohm}{type=symbols, name={\ensuremath{\Omega}}, sort=Ohm, symbol={\ensuremath{\Omega}}, description={unit of electrical resistance}}
\newglossaryentry{angstrom}{type=symbols, name={\AA}, sort=angström, symbol={\AA}, description={non-SI unit of length}}
%
%\FullGlossaryEntry{zhaw}{main}%
%{ZHaW}{}%
%{Zürcher Hochschule der angewandten Wissenschaften}{}%
%{Name der Hochschule meines Vertrauens.}
%
%\FullGlossaryEntry{mnmt1}{main}%
%{MNMT1}{}%
%{Mathematik: Numerik für Maschinentechnik 1}{}%
%{Erstes Numerik-Modul im Studiengang Maschinentechnik an der \gls{zhaw}. Beinhaltet Taylor- und Fourier-Reihen, Numerik gewöhnlicher Differentialgleichungen anhand des Euler-, Taylor- und Runge-Kutta-Verfahren, Numerik nichtlinearer Differentialgleichungen, Lagrange- und Newton-Interpolation, Splines und Ausgleichsrechnung.}
%
%\FullGlossaryEntry{mnmt2}{main}%
%{MNMT2}{}%
%{Mathematik: Numerik für Maschinentechnik 2}{}%
%{Zweites Numerik-Modul im Studiengang Maschinentechnik an der \gls{zhaw}. Beinhaltet Randwertprobleme, numerische Differentiation und Integration, Numerik partieller Differentialgleichungen anhand der \gls{fdm} und \gls{fem}, Stabilität von numerischen Verfahren zur Lösung von gewöhnlichen Differentialgleichungen, Schrittweitensteuerung und implizite Runge-Kutta-Verfahren.}
%
%\FullGlossaryEntry{fdm}{main}%
%{FDM}{}%
%{Finite Differenzen Methode}{}%
%{Numerisches Verfahren zur Lösung von \glspl{rwp}}
%
%\FullGlossaryEntry{fem}{main}%
%{FEM}{}%
%{Finite Elemente Methode}{}%
%{Numerisches Verfahren zur Lösung von \glspl{rwp}}
%
%\FullGlossaryEntry{rwp}{main}%
%{RWP}{RWPs}%
%{Randwertproblem}{Randwertprobleme}%
%{Ein Differentialproblem, bei dem diskrete Funktionswerte einer bestimmten Ordnung an beliebigen Stellen gegeben sind.}
%
%\FullGlossaryEntry{awp}{main}%
%{AWP}{AWPs}%
%{Anfangswertproblem}{Anfangswertprobleme}%
%{Ein Differentialproblem, bei dem alle diskrete Funktionswerte an einer Anfangsstelle gegeben sind.}
%
%\FullGlossaryEntry{dgl}{main}%
%{DGL}{DGLn}%
%{Differentialgleichung}{Differentialgleichungen}%
%{Eine Gleichung, bei der Ableitungen beliebiger Ordnung einer gesuchten Funktion vorkommen.}
%
%\FullGlossaryEntry{ode}{main}%
%{ode}{ode's}%
%{ordinary differential equation}{ordinary differential equations}%
%{Englische Bezeichnung für \gls{dgl}}
%
%
%\FullGlossaryEntry{dgls}{main}%
%{DGLS}{DGLSe}%
%{Differentialgleichungssystem}{Differentialgleichungssysteme}%
%{Ein System von Differentialgleichungen}
%
%\FullGlossaryEntry{ods}{main}%
%{ods}{ods's}%
%{ordinary differential system}{ordinary differential systems}%
%{Englische Bezeichnung für \gls{dgls}}
\fi

\logofilename{images/logos/SoE/de/de-soe-cmyk.png}
\projecttype{PA}
\major{HS16 Studiengang Informatik}
\title{[Trip]od}
\shorttitle{[Trip]od}
\author{Gruppe 15}
\authors{Gruppe 15: Fabio Costi, Raphael Emberger,\\Nicolas Kaiser, Julian Visser}
\mainreferee{Daniel Liebhart}
\referee{Liby Kunthrayil}
\industrypartner{}
\extreferee{}
\setdate{\today}

\newcommand{\AddRequirement}[2]{
\textbf{/#1#2/}
}
\newcommand{\F}[1]{
\AddRequirement{F1.}{#1}
}
\newcommand{\W}[1]{
\AddRequirement{F2.}{#1}
}
\newcommand{\D}[1]{
\AddRequirement{F3.}{#1}
}
\newcommand{\R}[1]{
\AddRequirement{F4.}{#1}
}
\newcommand{\TF}[1]{
\AddRequirement{T.}{#1}
}
\newcommand{\tableheader}[2]{\multicolumn{#1}{c}{\raisebox{-0.3em}[0ex][0ex]{\large{\textbf{#2}}}}}

\begin{document}

\maketitle

\tableofcontents

\chapter{Einleitung}\label{chp:Introduction}
Geplant ist ein Ferienplaner namens "[Trip]od".\\
Der Ferienplaner dient zur Verwaltung und effizienten Planung von Ferien für Einzelpersonen oder ganzen Gruppen. Die Applikation dient also zur automatischen Datensammlung relevanter Daten um möglichst kostengünstig die Ferien geniessen zu können. Dazu wird es eine Webseite als Schnittstellenplattform geben, welche die gesamte Verwaltung übernimmt. Auf dem Server läuft dann die gesamte Logik wie API-Aufrufe, Datenverwaltung, Filterung oder Ausgabe.
\chapter{Zielbestimmungen}\label{chp:DefinitionOfGoals}
\section{Musskriterien}\label{sec:MustCriteria}
In den folgenden Tabellen sind die Musskriterien des [Trip]od's definiert.
\begin{table}[ht]\centering
\begin{tabular}{l}\hline
\tableheader{1}{Account verwalten}\\[0.3em]\hline
\textbf{Kriterium}\\\hline
Account anlegen\\\hline
Account bearbeiten (Passwort ändern, persönliche Daten ausfüllen)\\\hline
Benutzer an- und abmelden\\\hline
\end{tabular}
\caption{Musskriterien Account}\label{tbl:MustAccount}
\end{table}

\begin{table}[ht]\centering
\begin{tabular}{l}\hline
\tableheader{1}{Ferien verwalten}\\[0.3em]\hline
\textbf{Kriterium}\\\hline
Ferien anlegen\\\hline
Personen hinzufügen und entfernen\\\hline
Reisedaten festlegen\\\hline
Budget festlegen\\\hline
Destination hinzufügen, bearbeiten, entfernen \\\hline
Verkehrsmittel hinzufügen, bearbeiten, entfernen\\\hline
Unterkunft hinzufügen, bearbeiten, entfernen\\\hline
Kommentar-Funktion\\\hline
Offline-Export\\\hline
\end{tabular}
\caption{Musskriterien Ferien}\label{tbl:MustVacation}
\end{table}\newpage
\section{Wunschkriterien}\label{sec:WishCriteria}
Um den Funktionsumfang bei verbleibender Zeit zu erweitern, wurden die in der folgenden Tabelle stehenden Wunschkriterien definiert.
\begin{table}[ht]\centering
\begin{tabular}{l}\hline
\tableheader{1}{Ferien verwalten}\\[0.3em]\hline
\textbf{Kriterium}\\\hline
Dokumente anhängen\\\hline
Mehrere Destinationen\\\hline
API-basierte Flugsuche\\\hline
API-basierte Unterkunftssuche\\\hline
\end{tabular}
\caption{Wunschkriterien Ferien}\label{tbl:WishVacation}
\end{table}
\section{Abgrenzungskriterien}\label{sec:DistinctionCriteria}
Die in diesem Kapitel definierten Abgrenzungskriterien sollen bewusst nicht erreicht werden.
\begin{table}[ht]\centering
\begin{tabular}{l|p{2.5cm}|p{5cm}|p{2cm}|p{1.5cm}}\hline
\tableheader{5}{Allgemein}\\[0.3em]\hline
\textbf{Kriterium} & \textbf{Kurzbeschrieb} & \textbf{Beschreibung} & \textbf{Komplexität} & \textbf{Priorität}\\\hline
\D{1000} & Booking & Die tatsächliche Tätigung von Käufen oder Buchungen soll nicht unterstützt werden. & hoch & tief\\\hline
\end{tabular}
\caption{Abgrenzungskriterien Allgemein}\label{tbl:DistinctionGeneral}
\end{table}
\chapter{Produkteinsatz}\label{chp:ProductApplication}
\section{Anwendungsbereiche}\label{sec:FieldOfApplience}
Genutzt wird die Applikation über das Web, welches die Zusammenarbeit von mehreren Personen an der gleichen Reise ermöglicht.
\section{Benutzergruppen des Produktes}\label{sec:TargetAudience}
Das Produkt gilt der öffentlichen, nicht-kommerziellen Nutzung und ist für Gruppen wie auch Einzelpersonen geeignet. Das umfasst Personen jeden Geschlechtes, Alters oder Region (Mehrsprachigkeit ist für die Seite vorerst jedoch nicht vorgesehen). Es sollen Personen mit Reiselust angesprochen werden, welche aber vielleicht kein grosses Budget zur Verfügung haben, wie Studenten. Das Programm soll auch eine Möglichkeit der Reiseplanung für Firmen oder Vereine darstellen.
\chapter{Funktionale Anforderungen}\label{chp:FunctionalRequirements}
Der Ferienplaner ist über einen eigenen Account zu verwalten. Jeder Reisende braucht einen eigenen Account (Dummy-Accounts für Gruppen könnten eine Alternative sein). Mit diesem Account lassen sich neue Ferien anlegen, denen andere Accounts hinzugefügt werden können. Bei Gruppen ist der Gruppenersteller automatisch der Gruppenadmin. Ferien können nun terminlich eingeschränkt, Abreise- und Rückkehrdaten festgelegt und Reiseziele bestimmt werden. Weitere Einstellmöglichkeiten (Minimum-Budget, Maximum-Komfort, etc.) können ebenfalls bestimmt werden. All diese Parameter können vom Admin uneingeschränkt und von Gruppenmitgliedern nach Revision des Admins und anderer Gruppenmitglieder aufgenommen werden.\\
Nun kann auf Knopfdruck die Berechnung gestartet werden. Dabei werden über die API's von mehreren Anbietern Angebote gesammelt und Rechnungen getätigt. Der Output soll aus den Kosten der Ideallösung (Auswahl der Möglichkeiten evtl. ebenfalls möglich), einer Tabelle der buchbaren Mittel und einem Zeitplan bestehen. Das effektive Buchen geht dann aus Sicherheitsgründen weiterhin händisch über den Admin.

\begin{table}[ht]\centering
\begin{tabular}{l|p{2.5cm}|p{5cm}|p{2cm}|p{1.5cm}}\hline
\tableheader{5}{Account verwalten}\\[0.3em]\hline
\textbf{Kriterium} & \textbf{Kurzbeschrieb} & \textbf{Beschreibung} & \textbf{Komplexität} & \textbf{Priorität}\\\hline
\F{1000} & Account anlegen & Es kann ein Account mit E-Mail, Passwort und persönlichen Daten erstellt werden. & mittel & hoch\\\hline
\F{1010} & Account bearbeiten & Das Passwort und die persönlichen Daten können angepasst werden. & tief & hoch\\\hline
\F{1020} & Benutzer anmelden & Ein Benutzer kann sich mit seinem Account (E-Mail und Passwort) anmelden. & mittel & hoch\\\hline
\F{1030} & Benutzer abmelden & Ein Benutzer kann sich, wenn er angemeldet ist, auch wieder abmelden. & tief & hoch\\\hline
\end{tabular}
\caption{Funktionale Anforderungen Account}\label{tbl:FuncAccount}
\end{table}

\begin{table}[ht]\centering
\begin{tabular}{l|p{2.5cm}|p{5cm}|p{2cm}|p{1.5cm}}\hline
\tableheader{5}{Ferien verwalten}\\[0.3em]\hline
\textbf{Kriterium} & \textbf{Kurzbeschrieb} & \textbf{Beschreibung} & \textbf{Komplexität} & \textbf{Priorität}\\\hline
\F{2000} & Ferien anlegen & Jede Person kann Ferien erstellen. & tief & hoch\\\hline
\F{2010} & Personen hinzufügen & Jede Person kann zusätzliche Personen hinzufügen. & tief & hoch\\\hline
\F{2020} & Personen entfernen & Jede Person kann sich oder andere Personen entfernen. & tief & hoch\\\hline
\F{2030} & Reisedaten festlegen & Der Zeitraum der Ferien kann festgelegt werden. & tief & hoch\\\hline
\F{2040} & Budget festlegen & Das Budget muss festgelegt werden. Dieses besteht aus den Kosten der Reise und der Unterkunft. & tief & hoch\\\hline
\F{2050} & Destination hinzufügen & Jede Person kann eine Destination hinzufügen. & tief & hoch\\\hline
\F{2060} & Destination bearbeiten & Jede Person kann eine Destination bearbeiten. & tief & hoch\\\hline
\F{2070} & Destination entfernen & Jede Person kann eine Destination entfernen. & tief & hoch\\\hline
\F{2080} & Verkehrsmittel hinzufügen & Jede Person kann ein Verkehrsmittel hinzufügen. Hierzu kann ein Link des jeweiligen Angebotes hinterlegt werden. & tief & hoch\\\hline
\F{2090} & Verkehrsmittel bearbeiten & Jede Person kann ein Verkehrsmittel bearbeiten. & tief & hoch\\\hline
\F{2100} & Verkehrsmittel entfernen & Jede Person kann ein Verkehrsmittel entfernen. & tief & hoch\\\hline
\F{2110} & Unterkunft hinzufügen & Jede Person kann eine Unterkunft hinzufügen. Hierzu kann ein Link des jeweiligen Angebotes hinterlegt werden. & tief & hoch\\\hline
\F{2120} & Unterkunft bearbeiten & Jede Person kann eine Unterkunft bearbeiten. & tief & hoch\\\hline
\F{2130} & Unterkunft entfernen & Jede Person kann eine Unterkunft entfernen. & tief & hoch\\\hline
\F{2140} & Kommentar-Funktion & Jede Person kann Kommentare hinterlassen (Chat). & mittel & hoch\\\hline
\F{2150} & Offline-Export & Jede Person kann einen Export mit den wichtigsten Daten der Ferien herunterladen. & mittel & hoch\\\hline
\W{2000} & Dokumente anhängen & Booking-Bestätigungen und ähnliche Dokumente sollen abgelegt werden können. & hoch & tief\\\hline
\W{2010} & Mehrere Destinationen & Es soll möglich sein, eine Planung für mehrere Destinationen hintereinander zu machen. & hoch & tief\\\hline
\W{2020} & API-basierte Flugsuche & Es soll möglich sein, Flugangebote über eine API abzurufen. z.B. Google Flights & hoch & tief\\\hline
\W{2030} & API-basierte Unterkunftssuche & Es soll möglich sein, Unterkünfte über eine API abzurufen. z.B. Booking.com & hoch & tief\\\hline
\end{tabular}
\caption{Funktionale Anforderungen Ferien}\label{tbl:FuncVacation}
\end{table}\newpage

\chapter{Funktionsbaum}\label{chp:FunctionTree}
\begin{minipage}{\textwidth}
%\begin{landscape}
\definecolor{MainBack1}{RGB}{66,66,68}
\definecolor{MainBack2}{RGB}{86,86,88}
\definecolor{MainFront}{RGB}{220,220,220}
\definecolor{SubBack1}{RGB}{240,240,240}
\definecolor{SubBack2}{RGB}{255,255,255}
\definecolor{SubFront}{RGB}{46,46,48}
\newcommand{\NewMainNode}[3]{ %NewNodeName, Pos, Text
\node(#1)[MainNode] at (#2) {#3};
}
\newcommand{\AddMainNode}[5]{ %NewNodeName, Parent, XOff, YOff, Text
\node(#1)[MainNode,below=of #2,shift={(#3, #4)}] {#5};
\coordinate(Temp) at ($(#1)-(#3, -1.3cm)$);
\draw[line](#2) -- (Temp) -| (#1);
}
\newcommand{\CreateNodeOffset}[2]{ %NewNodeName, Parent
\path (#2) + (30pt, -20.5pt) coordinate (#1);
}
\newcommand{\NewSubNode}[3]{ %NewNodeName, Parent, Text
\node(#1)[SubNode,below=of #2,shift={(40pt, 0pt)}] {#3};
\draw[line](#2) ++ (SubOffset) |- (#1);
}
\newcommand{\AddSubNode}[4]{ %NewNodeName, Parent, BelowSubNode Text
\node(#1)[SubNode,below=of #3] {#4};
\draw[line](#2) ++ (SubOffset) |- (#1);
}
\begin{turn}{90}
\begin{tikzpicture}[
text height = 0.1cm,
every node/.style = {
	column sep = 0.5cm,
	row sep = 1cm,
	thick,
	node distance = 1em,
	align = center,
	font = \small
},
base/.style = {
	text centered,
	inner sep = 1pt,
	minimum height = 30pt,
	minimum width = 100pt,
},
MainNode/.style = {
	base,
	rectangle,
	bottom color = MainBack1,
	top color = MainBack2,
	text = MainFront,
	draw = MainBack2
},
SubNode/.style = {
	base,
	rectangle,
	bottom color = SubBack1,
	top color = SubBack2,
	text = SubFront,
	draw = SubFront
},
line/.style = {
	draw,
	-stealth,
	shorten > = 2pt,
	thick
}
]

\coordinate(SubOffset) at (-30pt,-15pt);
\coordinate(MainOffset) at (12cm, 0cm);

%% LAYOUT
\NewMainNode{F2}{0, 0}{Ferien verwalten}

\NewSubNode{F2000}{F2}{Ferien anlegen}
\AddSubNode{F20100}{F2}{F2000}{Personen verwalten}

\node(F2010)[SubNode,right=of F20100,shift={(1cm, 0pt)}] {Personen hinzufügen};
\draw[line](F20100) |- (F2010);

\node(F2020)[SubNode,below=of F2010] {Personen abmelden};
\draw[line](F20100) ++ (2.5cm, 0pt) |- (F2020);

\AddSubNode{F2020}{F2}{F20100}{Daten festlegen}
\AddSubNode{F2030}{F2}{F2020}{Budget festlegen}
\AddSubNode{F20400}{F2}{F2030}{Destinationen verwalten}

\node(F2040)[SubNode,right=of F20400,shift={(11cm, 0pt)}] {Destinationen hinzufügen};
\draw[line](F20400) |- (F2040);
\node(F2050)[SubNode,below=of F2040] {Destinationen bearbeiten};
\draw[line](F20400) ++ (12.5cm, 0pt) |- (F2050);
\node(F2060)[SubNode,below=of F2050] {Destinationen entfernen};
\draw[line](F20400) ++ (12.5cm, 0pt) |- (F2060);

\AddSubNode{F20700}{F2}{F20400}{Verkehrsmittel verwalten}

\node(F2070)[SubNode,right=of F20700,shift={(6cm, 0pt)}] {Verkehrsmittel hinzufügen};
\draw[line](F20700) |- (F2070);
\node(F2080)[SubNode,below=of F2070] {Verkehrsmittel bearbeiten};
\draw[line](F20700) ++ (7.5cm, 0pt) |- (F2080);
\node(F2090)[SubNode,below=of F2080] {Verkehrsmittel entfernen};
\draw[line](F20700) ++ (7.5cm, 0pt) |- (F2090);

\AddSubNode{F21000}{F2}{F20700}{Unterkunft verwalten}

\node(F2100)[SubNode,right=of F21000,shift={(1cm, 0pt)}] {Unterkunft hinzufügen};
\draw[line](F21000) |- (F2100);
\node(F2110)[SubNode,below=of F2100] {Unterkunft bearbeiten};
\draw[line](F21000) ++ (2.5cm, 0pt) |- (F2110);
\node(F2120)[SubNode,below=of F2110] {Unterkunft entfernen};
\draw[line](F21000) ++ (2.5cm, 0pt) |- (F2120);

\AddSubNode{F2130}{F2}{F21000}{Kommentar-Funktion}
\AddSubNode{F2140}{F2}{F2130}{Dokumente anhängen}


\NewMainNode{F1}{$(F2)+(MainOffset)$}{Account}
\NewSubNode{F1010}{F1}{Account anlegen}
\AddSubNode{F1020}{F1}{F1010}{Account bearbeiten}

\end{tikzpicture}
\end{turn}
%\end{landscape}
\end{minipage}
\chapter{Nicht-Funktionale Anforderungen}\label{chp:NonFunctionalRequirements}
\begin{table}[ht]\centering
\begin{tabular}{l|p{4cm}|p{8cm}}\hline
\textbf{Kriterium} & \textbf{Kurzbeschrieb} & \textbf{Beschreibung}\\\hline
\R{1000} & Gebrauchsfähigkeit (Usability) & Die Webseiten müssen durch den Benutzer, welcher dem Benutzerprofil entspricht, ohne weitere Hilfe verwendet werden können.\\
\R{1010} & Fehlertoleranz & Hinweise und Fehlermeldungen müssen für den Benutzer verständlich formuliert sein und eine Hilfestellung zur Problemlösung beinhalten.\\
\R{1020} & Sprache \& länderspezifische Einstellungen & Die Webseiten sind in deutscher Sprache (Schweiz) verfasst, verwenden den Zeichensatz UTF-8 und die Schweiz-spezifischen Einstellungen von Datum, Zeit, Zahlen und Währung.\\
\R{1030} & Zielplattform (Server) & Die Web-Applikation muss als JavaServer-Pages auf dem zur Verfügung gestellten virtuellen Server unter Verwendung einer SQL-Datenbank mit Apache Tomcat betrieben werden.\\
\R{1040} & Zielplattform (Client) & Die Webseiten werden in der aktuellsten freigegebenen Version des Mozilla Firefox und Google Chrome korrekt dargestellt.\\
\R{1050} & Werkzeuge zur Entwicklung & Als Projektmanagement-Tool und zur Verwaltung des Sourcecode und der Dokumente muss der zur Verfügung gestellte github Server verwendet werden.\\
\R{1060} & Robustheit & Auch nach einem Neustart des virtuellen Servers muss die Webseite voll funktionsfähig sein.\\
\R{1070} & Testbarkeit & Für die Durchführung der Tests und der Abnahme müssen sinnvolle Testdaten in genügendem Umfang zur Verfügung gestellt werden.\\
\end{tabular}
\caption{Nicht-Funktionale Anforderungen}\label{tbl:NonFunctionalRequirements}
\end{table}
\chapter{Abnahmekriterien Anforderungen}\label{chp:TestRequirements}
\begin{table}[ht]\centering
\begin{tabular}{l|p{4cm}|p{8cm}}\hline
Kriterium & Kurzbeschrieb & Beschreibung\\\hline
\TF{1000} & Account anlegen und Profil befüllen & Es ist ein Account anzulegen mit dem Namen "Test Account" und der E-Mail-Addresse "account@test.me". Weiter soll das Profil wie folgt befüllt werden:\ldots
\end{tabular}
\caption{Abnahmekriterien}\label{tbl:TestRequirements}
\end{table}
\chapter{Verzeichnisse}\label{chp:Index}
\printglossary\label{sec:Glossar}
\newpage
\bibliography{reference}\label{sec:Bibliography}
\end{document}