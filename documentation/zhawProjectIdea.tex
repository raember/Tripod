%%	build-queue:
%%	
%%	xelatex
%%	makeglossaries
%%	makeindex
%%	xelatex
%%	bibtex
%%	xelatex
%%	xelatex
%%	

\RequirePackage[l2tabu,orthodox]{nag}
\documentclass[10pt,a4paper,titlepage,twoside,german]{zhawreprt}

\include{packages}
\if false
%\newcommand{\FullGlossaryEntry}[7]{%Abkz., Typ, Name, Name(plural), Beschr., Beschr.(plural), Erkl.
%\newglossaryentry{#1}{type=\acronymtype,
%name={\textit{#3}},
%plural={\textit{#4}},
%description={\textit{#5}},
%first={\textit{#5}(nachfolgend #3)\glsadd{_#1}},
%firstplural={\textit{#6}(nachfolgend \glsentryplural{#4})},
%see=[Glossar:]{_#1}
%}
%\longnewglossaryentry{_#1}{type=#2,
%name={\textit{#5}},
%plural={\textit{#6}}
%}{\hspace{0pt}\\#7}
%}


% Abkuerzungen
\newacronym{jlu}{JLU}{Justus-Liebig-Universität}
\newacronym{hrz}{HRZ}{Hochschulrechenzentrum}
\newacronym[plural=LEDs, longplural={light-emitting diodes}]{led}{LED}{light-emitting diode}
\newacronym[plural=EEPROMs, longplural={electrically erasable programmable read-only memories}]{eeprom}{EEPROM}{electrically erasable programmable read-only memory}

%  Glossareintraege
\newglossaryentry{culdesac}{name=cul-de-sac, description={passage or street closed at one end}, plural=culs-de-sac}
\newglossaryentry{elite}{name={é}lite, description={select group or class}, sort=elite}
\newglossaryentry{elitism}{name={é}litism, description={advocacy of dominance by an \gls{elite}}, sort=elitism}
\newglossaryentry{attache}{name=attaché, description={person with special diplomatic responsibilities}}

% Eintraege für Symbolliste
\newglossaryentry{ohm}{type=symbols, name={\ensuremath{\Omega}}, sort=Ohm, symbol={\ensuremath{\Omega}}, description={unit of electrical resistance}}
\newglossaryentry{angstrom}{type=symbols, name={\AA}, sort=angström, symbol={\AA}, description={non-SI unit of length}}




%%%%%%%%%%%%%%%% TRIPOD %%%%%%%%%%%%%%%%
\newglossaryentry{tripod}{name=[Trip]od, description={Name der zu entwickelnden Applikation}}
\newglossaryentry{api}{name=API, description={\textit{Application Programming Interface}, Schnittstelle einer Applikation zur Nutzung über andere Programme}}
\fi

\logofilename{images/logos/SoE/de/de-soe-cmyk.png}
\projecttype{PA}
\major{HS16 Studiengang Informatik}
\title{Vacationplanner}
\shorttitle{Vacationplanner}
\author{Gruppe 15}
\authors{Gruppe 15: Fabio Costi, Raphael Emberger,\\Nicolas Kaiser, Julian Visser}
\mainreferee{Daniel Liebhart}
\referee{Liby Kunthrayil}
\industrypartner{}
\extreferee{}
\setdate{\today}

\begin{document}

\maketitle

\tableofcontents

\chapter{Ziel}\label{chp:Objective}
\notes{\item Ziel der Webseite(z.B. Information, Spass, Lernplattform, Chatroom, Werbung)}
\chapter{Produkteinsatz}\label{chp:FieldOfApplience}
\notes{
\item Anwendungsbereich(z.B. kommerzielle Nutzung, private Nutzung)
\item Zielgruppen der Webseite(z.B. Alter, Geschlecht, Region etc.;z.B. Studierende aller Fachhochschulen in der Schweiz, Studierende der ZHAW, Studierende der School of Engineering, Studierende einer Klasse)
}
\chapter{Produktfunktionen}\label{chp:ProductFunctions}
\notes{
\item Nennung der wichtigsten Arbeitsabläufe, die mit der Webseite durchgeführt werden können
\item Weitere Ideen/Abgrenzungen(was kann die erste Version noch nicht; was wären weitere Funktionen für nächste Versionen etc.)
}
\chapter{Gruppenvorkenntnisse}\label{chp:KnowHow}
\begin{table}[h!]\label{tbl:KnowHow}
\begin{center}
\begin{tabular}{|r|l|}
\hline
\textbf{Name} & \textbf{Vorkenntnisse}\\\hline\hline
Raphael Emberger &
\parbox[t]{11cm}{Ausbildung: Konstrukteur(2+ Jahre)\\
Programmiererfahrung: Autodidaktisch seit 2012\\
Programmierkenntnisse: \emph{VB.NET}, \emph{C\#}, \emph{VBA}, \emph{VBScript}, \emph{Matlab}, \emph{Lua}, \emph{C}, \emph{C++}\\
Webtechnologiekenntnisse: relativ erfahren\\
Projekterfahrung: An Maschinenbau-Projekten(Studium, Industrie) mitgearbeitet oder geleitet.}\\
\hline
\end{tabular}\caption{Grundvorkenntnisse}
\end{center}
\end{table}
\notes{
\item Beschreiben Sie Ihren Hintergrund (Ausbildung, Berufserfahrung), Programmiererfahrung(zeitlich und inhaltlich), Erfahrung bezüglich Webtechnologien(zeitlich und inhaltlich) und Projekterfahrung(zeitlich, Umfang, Methodik, Rolle im Team).
\item Diese Angaben verlangen wir individuell für jedes einzelne Gruppenmitglied. Sie sind wichtig, um abschätzen zu können, ob die Umsetzung Ihrer Projektidee realistisch ist.
}
\chapter{Anhang: Dokumentation Brainstorming}\label{chp:AppendixBrainstorming}
\notes{
\item Liste der fünf wichtigsten „Ideengruppen“ mit ein bis zwei erklärenden Sätzen pro „Ideengruppe“
\item Fünf Bewertungskriterien (Tabelle mit Kriterien, Skala pro Kriterium, erklärendem Satz pro Kriterium, allfälligen Gewichtungen, Begründungen der Gewichtungen)
\item Übersicht Bewertung der Projektideen (Tabelle, ohne Kommentare)
}
\end{document}
