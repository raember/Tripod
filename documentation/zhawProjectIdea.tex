%%	build-queue:
%%	
%%	xelatex
%%	makeglossaries
%%	makeindex
%%	bibtex
%%	xelatex
%%	xelatex
%%	

\RequirePackage[l2tabu,orthodox]{nag}
\documentclass[10pt,a4paper,titlepage,twoside,german,final]{zhawreprt}

	\setdefaultlanguage{german}
\RequirePackage{lscape}
\usepackage{pgfgantt}
\usepackage{xcolor}
	\definecolor{lgray}{RGB}{250,250,250}
	\definecolor{lgreen}{RGB}{63,127,95}
	\definecolor{lred}{RGB}{127,0,85}
	\definecolor{lblue}{RGB}{42,0,255}
\usepackage{listings}
	\lstdefinestyle{basestyle}{
		basicstyle=\small\ttfamily,
		breakatwhitespace = true,
		tabsize = 4,
		frame = double,
		numbers = left,
		numbersep = 10pt,
		numberstyle = {\tiny\emptyaccsupp},
%		firstnumber = auto,
		numberblanklines = true,
		captionpos = b,
		columns = fullflexible,
		extendedchars = true,
		float = ht,
		showtabs = false,
		showspaces=false,
		showstringspaces=false,
		breaklines=true,
%		prebreak=\Righttorque
		backgroundcolor=\color{lgray},
		keywordstyle=\color{lred}\bfseries,
		commentstyle=\color{lgreen}\ttfamily,
%		morekeywords={printstr, printhexln},
		stringstyle=\color{lblue},
		xleftmargin = \fboxsep,
		xrightmargin = -6pt,
		showstringspaces=true,
	}
	\newcommand{\setlistingCSharp}{
		\lstset{
		style = basestyle,
		language = [Sharp]C,
		%otherkeywords = {*,<,>,=,;,\{,\}},
		%deletekeywords = {...},
	}}
	\newcommand{\setlistingCpp}{
		\lstset{
		style = basestyle,
		language = C++,
		%otherkeywords = {*,<,>,=,;,\{,\}},
		%deletekeywords = {...},
	}}
	\newcommand{\setlistingJava}{
		\lstset{
		style = basestyle,
		language = Java,
		%otherkeywords = {*,<,>,=,;,\{,\}},
		%deletekeywords = {...},
	}}
	\newcommand{\setlistingLaTeX}{
		\lstset{
		style = basestyle,
		language = TeX,
		%otherkeywords = {*,<,>,=,;,\{,\}},
		%deletekeywords = {...},
	}}
	\newcommand{\setlistingMatlab}{
		\lstset{
		style = basestyle,
		language = Matlab,
		otherkeywords = {methods,enumeration,properties,classdef,Sealed,Abstract},
		%deletekeywords = {...},
	}}
\usepackage{xstring}
	\newcommand{\inlist}[2]{
		\IfSubStr{,#2,}{,\arabic{#1},}{\color{lgray!95!blue}}{\color{lgray}}
	}
\usepackage{lstlinebgrd}
	\makeatletter
	\renewcommand{\lst@linebgrd}{%
	\ifx\lst@linebgrdcolor\empty\else
		\rlap{%
			\lst@basicstyle
			\color{lgray}
			\lst@linebgrdcolor{%
				\kern-\dimexpr\lst@linebgrdsep\relax%
				\lst@linebgrdcmd{\lst@linebgrdwidth}{\lst@linebgrdheight}{\lst@linebgrddepth }%
			}%
		}%
	\fi}
	\makeatother
\usepackage[space=true]{accsupp}
	\newcommand\emptyaccsupp[1]{\BeginAccSupp{ActualText={}}#1\EndAccSupp{}}
\usepackage{rotating}
\usepackage{mathptmx}
\usepackage{amssymb}
\usepackage{textcomp}
\usepackage[squaren]{SIunits}
\usepackage{amsmath}
\usepackage{amsfonts}
\usepackage{amssymb}
\usepackage[toc,page]{appendix}
\usepackage{fontspec}
	\setmainfont{Arial} % sets the roman font
	\setsansfont{Arial} % sets the sans-sérif font
	\setmonofont{Arial} % sets the monospace font
\usepackage{microtype}
\usepackage{colortbl}
\usepackage{tabularx}
\usepackage{longtable}
\usepackage{pgf,tikz}
	\usetikzlibrary{shapes}
	\usetikzlibrary{shapes.geometric}
	\usetikzlibrary{shapes.arrows}
	\usetikzlibrary{positioning}
	\usetikzlibrary{fit}
	\usetikzlibrary{calc}
	\usetikzlibrary{patterns}
\usepackage{pgfplots}
	\pgfplotsset{compat=1.13}
\usepackage{array}
\usepackage{natbib}
	\bibliographystyle{agsm}
\usepackage{usecases}
\usepackage{footnote}
	\makesavenoteenv{description}
\usepackage{makeidx}
	\makeindex
\usepackage[nottoc]{tocbibind}
\makeatletter
	\if@inltxdoc\else
	  \renewenvironment{theindex}%
	    {\if@twocolumn
	       \@restonecolfalse
	     \else
	       \@restonecoltrue
	     \fi
	     \if@bibchapter
	        \if@donumindex
	          \refstepcounter{section}
	          \twocolumn[\vspace*{2\topskip}%
	                     \@makechapterhead{\indexname}]%
	          \addcontentsline{toc}{section}{\protect\numberline{\thesection}\indexname}
	          \sectionmark{\indexname}
	        \else
	          \if@dotocind
	            \twocolumn[\vspace*{2\topskip}%
	                       \@makeschapterhead{\indexname}]%
	            \prw@mkboth{\indexname}
	            \addcontentsline{toc}{section}{\protect\numberline{\thesection}\indexname}
	          \else
	            \twocolumn[\vspace*{2\topskip}%
	                       \@makeschapterhead{\indexname}]%
	            \prw@mkboth{\indexname}
	          \fi
	        \fi
	     \else
	        \if@donumindex
	          \twocolumn[\vspace*{-1.5\topskip}%
	                     \@nameuse{\@tocextra}{\indexname}]%
	          \csname \@tocextra mark\endcsname{\indexname}
	        \else
	          \if@dotocind
	            \twocolumn[\vspace*{-1.5\topskip}%
	                       \toc@headstar{\@tocextra}{\indexname}]%
	            \prw@mkboth{\indexname}
	            \addcontentsline{toc}{\@tocextra}{\indexname}
	          \else
	            \twocolumn[\vspace*{-1.5\topskip}%
	                       \toc@headstar{\@tocextra}{\indexname}]%
	            \prw@mkboth{\indexname}
	          \fi
	        \fi
	     \fi
	   \thispagestyle{plain}\parindent\z@
	   \parskip\z@ \@plus .3\p@\relax
	   \let\item\@idxitem}
	   {\if@restonecol\onecolumn\else\clearpage\fi}
	\fi
\makeatother
\usepackage{etoolbox}
\makeatletter
	\renewcommand\listoftables{%
	    \section{Tabellenverzeichnis}%
	    \@mkboth{\MakeUppercase\listtablename}%
	        {\MakeUppercase\listtablename}%
	    \@starttoc{lot}%
	}
	\renewcommand\listoffigures{%
	    \section{Abbildungsverzeichnis}%
	    \@mkboth{\MakeUppercase\listfigurename}%
	        {\MakeUppercase\listfigurename}%
	    \@starttoc{lof}%
	}
	\renewcommand\lstlistoflistings{%
	    \section{Listingverzeichnis}%
	    \@mkboth{\MakeUppercase\listfigurename}%
	        {\MakeUppercase\listfigurename}%
	    \@starttoc{lol}%
	}
	\renewenvironment{thebibliography}[1]
	     {\section{\bibname}% <-- this line was changed from \chapter* to \section*
	      \@mkboth{\MakeUppercase\bibname}{\MakeUppercase\bibname}%
	      \list{\@biblabel{\@arabic\c@enumiv}}%
	           {\settowidth\labelwidth{\@biblabel{#1}}%
	            \leftmargin\labelwidth
	            \advance\leftmargin\labelsep
	            \@openbib@code
	            \usecounter{enumiv}%
	            \let\p@enumiv\@empty
	            \renewcommand\theenumiv{\@arabic\c@enumiv}}%
	      \sloppy
	      \clubpenalty4000
	      \@clubpenalty \clubpenalty
	      \widowpenalty4000%
	      \sfcode`\.\@m}
	     {\def\@noitemerr
	       {\@latex@warning{Empty `thebibliography' environment}}%
	      \endlist}
\makeatother
	
	\renewcommand\appendixname{Appendix}
\makeatletter
	\renewenvironment{theindex}{
		\renameindex
		\let\ps@plainorig\ps@plain
		\let\ps@plain\ps@scrheadings
		\if@twocolumn
			\@restonecolfalse
		\else
			\@restonecoltrue
		\fi
		\columnseprule \z@
		\columnsep 35\p@
		\twocolumn[\section{\indexname}]%
		\@mkboth{\MakeUppercase\indexname}{\MakeUppercase\indexname}%
		\thispagestyle{plain}\parindent\z@
		\parskip\z@ \@plus .3\p@\relax
	\let\item\@idxitem}
	{\if@restonecol\onecolumn\else\clearpage\fi}
\makeatother

\usepackage{varioref}
\usepackage[colorlinks=true, linkcolor=black, citecolor=black, plainpages=false, unicode, pdfencoding=auto ,backref=page]{hyperref}
\usepackage{cleveref}
\usepackage[automake,acronym,numberedsection]{glossaries-extra}
	\renewcommand*{\glspostdescription}{}
	\newglossary[slg]{symbols}{sym}{sbl}{Symbolverzeichnis}
	\makeglossaries
	\loadglsentries{glossaryentries.tex}
	\pdfstringdefDisableCommands{\let\textenglish\@firstofone\let\textgerman\@firstofone}
	\makeatletter
	\renewcommand*{\@@glossarysec}{section}
	\makeatother
\usepackage{adjustbox}
\usepackage{array}
\usepackage{booktabs}
\usepackage{multirow}
\if false
%\newcommand{\FullGlossaryEntry}[7]{%Abkz., Typ, Name, Name(plural), Beschr., Beschr.(plural), Erkl.
%\newglossaryentry{#1}{type=\acronymtype,
%name={\textit{#3}},
%plural={\textit{#4}},
%description={\textit{#5}},
%first={\textit{#5}(nachfolgend #3)\glsadd{_#1}},
%firstplural={\textit{#6}(nachfolgend \glsentryplural{#4})},
%see=[Glossar:]{_#1}
%}
%\longnewglossaryentry{_#1}{type=#2,
%name={\textit{#5}},
%plural={\textit{#6}}
%}{\hspace{0pt}\\#7}
%}


% Abkuerzungen
\newacronym{jlu}{JLU}{Justus-Liebig-Universität}
\newacronym{hrz}{HRZ}{Hochschulrechenzentrum}
\newacronym[plural=LEDs, longplural={light-emitting diodes}]{led}{LED}{light-emitting diode}
\newacronym[plural=EEPROMs, longplural={electrically erasable programmable read-only memories}]{eeprom}{EEPROM}{electrically erasable programmable read-only memory}

%  Glossareintraege
\newglossaryentry{culdesac}{name=cul-de-sac, description={passage or street closed at one end}, plural=culs-de-sac}
\newglossaryentry{elite}{name={é}lite, description={select group or class}, sort=elite}
\newglossaryentry{elitism}{name={é}litism, description={advocacy of dominance by an \gls{elite}}, sort=elitism}
\newglossaryentry{attache}{name=attaché, description={person with special diplomatic responsibilities}}

% Eintraege für Symbolliste
\newglossaryentry{ohm}{type=symbols, name={\ensuremath{\Omega}}, sort=Ohm, symbol={\ensuremath{\Omega}}, description={unit of electrical resistance}}
\newglossaryentry{angstrom}{type=symbols, name={\AA}, sort=angström, symbol={\AA}, description={non-SI unit of length}}
%
%\FullGlossaryEntry{zhaw}{main}%
%{ZHaW}{}%
%{Zürcher Hochschule der angewandten Wissenschaften}{}%
%{Name der Hochschule meines Vertrauens.}
%
%\FullGlossaryEntry{mnmt1}{main}%
%{MNMT1}{}%
%{Mathematik: Numerik für Maschinentechnik 1}{}%
%{Erstes Numerik-Modul im Studiengang Maschinentechnik an der \gls{zhaw}. Beinhaltet Taylor- und Fourier-Reihen, Numerik gewöhnlicher Differentialgleichungen anhand des Euler-, Taylor- und Runge-Kutta-Verfahren, Numerik nichtlinearer Differentialgleichungen, Lagrange- und Newton-Interpolation, Splines und Ausgleichsrechnung.}
%
%\FullGlossaryEntry{mnmt2}{main}%
%{MNMT2}{}%
%{Mathematik: Numerik für Maschinentechnik 2}{}%
%{Zweites Numerik-Modul im Studiengang Maschinentechnik an der \gls{zhaw}. Beinhaltet Randwertprobleme, numerische Differentiation und Integration, Numerik partieller Differentialgleichungen anhand der \gls{fdm} und \gls{fem}, Stabilität von numerischen Verfahren zur Lösung von gewöhnlichen Differentialgleichungen, Schrittweitensteuerung und implizite Runge-Kutta-Verfahren.}
%
%\FullGlossaryEntry{fdm}{main}%
%{FDM}{}%
%{Finite Differenzen Methode}{}%
%{Numerisches Verfahren zur Lösung von \glspl{rwp}}
%
%\FullGlossaryEntry{fem}{main}%
%{FEM}{}%
%{Finite Elemente Methode}{}%
%{Numerisches Verfahren zur Lösung von \glspl{rwp}}
%
%\FullGlossaryEntry{rwp}{main}%
%{RWP}{RWPs}%
%{Randwertproblem}{Randwertprobleme}%
%{Ein Differentialproblem, bei dem diskrete Funktionswerte einer bestimmten Ordnung an beliebigen Stellen gegeben sind.}
%
%\FullGlossaryEntry{awp}{main}%
%{AWP}{AWPs}%
%{Anfangswertproblem}{Anfangswertprobleme}%
%{Ein Differentialproblem, bei dem alle diskrete Funktionswerte an einer Anfangsstelle gegeben sind.}
%
%\FullGlossaryEntry{dgl}{main}%
%{DGL}{DGLn}%
%{Differentialgleichung}{Differentialgleichungen}%
%{Eine Gleichung, bei der Ableitungen beliebiger Ordnung einer gesuchten Funktion vorkommen.}
%
%\FullGlossaryEntry{ode}{main}%
%{ode}{ode's}%
%{ordinary differential equation}{ordinary differential equations}%
%{Englische Bezeichnung für \gls{dgl}}
%
%
%\FullGlossaryEntry{dgls}{main}%
%{DGLS}{DGLSe}%
%{Differentialgleichungssystem}{Differentialgleichungssysteme}%
%{Ein System von Differentialgleichungen}
%
%\FullGlossaryEntry{ods}{main}%
%{ods}{ods's}%
%{ordinary differential system}{ordinary differential systems}%
%{Englische Bezeichnung für \gls{dgls}}
\fi

\logofilename{images/logos/SoE/de/de-soe-cmyk.png}
\projecttype{PA}
\major{HS16 Studiengang Informatik}
\title{[Trip]od}
\shorttitle{[Trip]od}
\author{Gruppe 15}
\authors{Gruppe 15: Fabio Costi, Raphael Emberger,\\Nicolas Kaiser, Julian Visser}
\mainreferee{Daniel Liebhart}
\referee{Liby Kunthrayil}
\industrypartner{}
\extreferee{}
\setdate{\today}

\begin{document}

\maketitle

\tableofcontents

\chapter{Ziel}\label{chp:Objective}
Geplant ist ein "Vacation planner" - ein Ferienplaner.\\
Der Ferienplaner dient zur Verwaltung und effizienten Planung von Ferien für Einzelpersonen oder ganzen Gruppen. Die Applikation dient also zur automatischen Datensammlung relevanter Daten um möglichst kostengünstig die Ferien geniessen zu können. Dazu wird es eine Webseite als Schnittstellenplattform geben, welche die gesamte Verwaltung übernimmt. Auf dem Server läuft dann die gesamte Logik wie API-Aufrufe, Datenverwaltung, Filterung oder Ausgabe.
\notes{\item Ziel der Webseite(z.B. Information, Spass, Lernplattform, Chatroom, Werbung)}
\chapter{Produkteinsatz}\label{chp:FieldOfApplience}
Das Produkt gilt der öffentlichen, nicht-kommerziellen Nutzung und ist Gruppen wie auch Einzelpersonen offen. Das umfasst Personen jeden Geschlechtes, Alter oder Region(Mehrsprachigkeit ist für die Seite vorerst jedoch nicht vorgesehen). Es sollen Personen mit Reiselust angesprochen werden, aber vielleicht kein grosses Budget zur Verfügung haben wie Studenten. Das Programm soll auch eine Möglichkeit der Reiseplanung für Firmen oder Vereinen darstellen.
\notes{
\item Anwendungsbereich(z.B. kommerzielle Nutzung, private Nutzung)
\item Zielgruppen der Webseite(z.B. Alter, Geschlecht, Region etc.;z.B. Studierende aller Fachhochschulen in der Schweiz, Studierende der ZHAW, Studierende der School of Engineering, Studierende einer Klasse)
}
\chapter{Produktfunktionen}\label{chp:ProductFunctions}
Der Ferienplaner ist über einen eigenen Account zu verwalten. Jeder Reisende braucht einen eigenen Account(Dummy-Accounts für Gruppen könnten eine Alternative sein). Mit diesem Account lassen sich neue Ferien anlegen, denen andere Accounts anknüpfen lassen. Bei Gruppen ist der Gruppenersteller der Gruppenadmin. Ferien können nun terminlich eingeschränkt, Abreise und Rückkehrdaten festgelegt und Reiseziele bestimmt werden. Weitere Einstellmöglichkeiten (Minimum-Budget, Maximum-Komfort, etc.) können ebenfalls bestimmt werden. All diese Parameter können vom Admin uneingeschränkt und von Gruppenmitgliedern nach Revision des Admins und anderer Gruppenmitglieder aufgenommen werden.\\
Nun kann auf Knopfdruck die Berechnung gestartet werden. Dabei wird über die API's von mehreren Anbietern Angebote gesammelt und Rechnung getätigt. Der Output soll aus Kosten der Ideallösung(Auswahl der Möglichkeiten evtl. ebenfalls möglich), einer Tabelle der buchbaren Mittel und Zeitplan sein. Das effektive Buchen geht dann aus Sicherheitsgründen weiterhin händisch über den Admin.
\notes{
\item Nennung der wichtigsten Arbeitsabläufe, die mit der Webseite durchgeführt werden können
\item Weitere Ideen/Abgrenzungen(was kann die erste Version noch nicht; was wären weitere Funktionen für nächste Versionen etc.)
}
\chapter{Gruppenvorkenntnisse}\label{chp:KnowHow}
\begin{table}[h!]
\begin{tabular}{|r|l|}
\hline
\textbf{Name} & \textbf{Vorkenntnisse}\\\hline\hline
Fabio Costi &
\parbox[t]{11cm}{\textbf{Ausbildung:}\\
\textbf{Programmiererfahrung:}\\
\textbf{Programmierkenntnisse:}\\
\textbf{Webtechnologiekenntnisse:}\\
\textbf{Projekterfahrung:}
}\\\hline
Raphael Emberger &
\parbox[t]{11cm}{\textbf{Ausbildung:} Konstrukteur(2+ Jahre)\\
\textbf{Programmiererfahrung:} Autodidaktisch seit 2012\\
\textbf{Programmierkenntnisse:} \emph{VB.NET}, \emph{C\#}, \emph{Java}, \emph{VBA}, \emph{VBScript}, \emph{Matlab}, \emph{Lua}, \emph{C}, \emph{C++}\\
\textbf{Webtechnologiekenntnisse:} relativ erfahren\\
\textbf{Projekterfahrung:} An Maschinenbau-Projekten(Studium, Industrie) mitgearbeitet oder geleitet.
}\\\hline
Nicolas Kaiser &
\parbox[t]{11cm}{\textbf{Ausbildung:}\\
\textbf{Programmiererfahrung:}\\
\textbf{Programmierkenntnisse:}\\
\textbf{Webtechnologiekenntnisse:}\\
\textbf{Projekterfahrung:}
}\\\hline
Julian Visser &
\parbox[t]{11cm}{\textbf{Ausbildung:}\\
\textbf{Programmiererfahrung:}\\
\textbf{Programmierkenntnisse:}\\
\textbf{Webtechnologiekenntnisse:}\\
\textbf{Projekterfahrung:}
}\\\hline
\end{tabular}\caption{Grundvorkenntnisse}\label{tbl:KnowHow}
\end{table}
\notes{
\item Beschreiben Sie Ihren Hintergrund (Ausbildung, Berufserfahrung), Programmiererfahrung (zeitlich und inhaltlich), Erfahrung bezüglich Webtechnologien(zeitlich und inhaltlich) und Projekterfahrung(zeitlich, Umfang, Methodik, Rolle im Team).
\item Diese Angaben verlangen wir individuell für jedes einzelne Gruppenmitglied. Sie sind wichtig, um abschätzen zu können, ob die Umsetzung Ihrer Projektidee realistisch ist.
}
\chapter{Anhang: Dokumentation Brainstorming}\label{chp:AppendixBrainstorming}
Die Ideen wurden in drei Gruppen unterteilt:
\begin{itemize}
\item Spiele\\
Spiele, die alleine, zu zweit oder mit mehreren Personen gespielt werden können.
\item Information\\
Applikationen, die spezifische Informationen liefern. Grappler, Filter, etc.
\item Sonstiges\\
Exoten, wie Simulatoren.
\end{itemize}
Die beim Brianstorming gesammelten Ideen wurden tabellarisiert, anhand von Kriterien bewertet und ihren kriteriumsspezifisch gewichteten Gesamtwert berechnet.
\begin{itemize}
\item Know-How\\
Der Wissensstand der Mitglieder bezüglich der Idee.\\
Wird mit 2 gewichtet, da das Wissen um den Sachverhalt massgebend ist.
\item Interesse\\
Wie gross ist das Interesse, die Idee umzusetzen.\\
Das Interesse wurde als wichtigstes Kriterium auserkoren(Gewichtung: 3), da die Motivation unter fehlendem Interesse leiden würde.
\item Umsetzbarkeit\\
Ist der Umfang des Programmes zumutbar?\\
Da es unerwünscht ist, das Projekt nicht beenden zu können, wird diese Kategorie mit 2 gewichtet.
\item Erweiterbarkeit\\
Existiert die Möglichkeit eines Ausbaus der Applikation?\\
Die Erweiterbarkeit wäre "nice-to-have" und wird deshalb mit 1 gewichtet.
\item Originalität\\
Gibt es schon viele ähnliche Lösungen?\\
Ob es bereits Lösungen dazu gibt, ist nicht von grossem Interesse. Der Lernefekt ist wichtiger. Deshalb wird diese Kategorie mit 1 gewichtet.
\end{itemize}
Damit wurden dann alle Ideen bewertet:
\begin{table}[!h]\centering
\begin{tabular}{l|c|c|c|c|c|c}
Name & \rotatebox{90}{Know-How} & \rotatebox{90}{Interesse} & \rotatebox{90}{Umsetzbarkeit} & \rotatebox{90}{Erweiterbarkeit} & \rotatebox{90}{Originalität} & $\sum$\\\hline
Gewichtung & 2 & 3 & 2 & 1 & 1 & \\\hline\hline
Ferienplanung & 1 & 1 & 2 & 2 & 2 & 13\\
Aufgabenheft & 2 & 1 & 2 & 1 & 0 & 12\\
Scripter & 1 & 1 & 2 & 2 & 1 & 12\\
Automat & 1 & 1 & 2 & 1 & 2 & 12\\
Flappy-Bird & 2 & 1 & 2 & 0 & -2 & 9\\
Pong & 2 & 0 & 2 & 2 & -2 & 8\\
Planeten-Simulation & 1 & 0 & 2 & 2 & 0 & 8\\
Mastermind & 1 & 0 & 2 & 1 & 0 & 7\\
Gitarrenakkord Sucher & 1 & -1 & 2 & 2 & 1 & 6\\
Online-Budgetplaner & 1 & 0 & 2 & 1 & -2 & 5\\
Notenübersicht & 2 & -1 & 2 & 1 & -2 & 4\\
Evolutions-Simulator & -1 & 0 & 1 & 2 & 2 & 4\\
ASVZ Besetzung & 1 & -1 & 1 & 0 & 0 & 1\\
Fitness-App & -2 & -1 & 1 & 2 & -2 & -5
\end{tabular}\caption{Ideen}\label{tbl:Ideas}
\end{table}
\notes{
\item Liste der fünf wichtigsten „Ideengruppen“ mit ein bis zwei erklärenden Sätzen pro „Ideengruppe“
\item Fünf Bewertungskriterien (Tabelle mit Kriterien, Skala pro Kriterium, erklärendem Satz pro Kriterium, allfälligen Gewichtungen, Begründungen der Gewichtungen)
\item Übersicht Bewertung der Projektideen (Tabelle, ohne Kommentare)
}
\end{document}
