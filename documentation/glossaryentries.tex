%\newcommand{\FullGlossaryEntry}[7]{%Abkz., Typ, Name, Name(plural), Beschr., Beschr.(plural), Erkl.
%\newglossaryentry{#1}{type=\acronymtype,
%name={\textit{#3}},
%plural={\textit{#4}},
%description={\textit{#5}},
%first={\textit{#5}(nachfolgend #3)\glsadd{_#1}},
%firstplural={\textit{#6}(nachfolgend \glsentryplural{#4})},
%see=[Glossar:]{_#1}
%}
%\longnewglossaryentry{_#1}{type=#2,
%name={\textit{#5}},
%plural={\textit{#6}}
%}{\hspace{0pt}\\#7}
%}


% Abkuerzungen
\newacronym{jlu}{JLU}{Justus-Liebig-Universität}
\newacronym{hrz}{HRZ}{Hochschulrechenzentrum}
\newacronym[plural=LEDs, longplural={light-emitting diodes}]{led}{LED}{light-emitting diode}
\newacronym[plural=EEPROMs, longplural={electrically erasable programmable read-only memories}]{eeprom}{EEPROM}{electrically erasable programmable read-only memory}

%  Glossareintraege
\newglossaryentry{culdesac}{name=cul-de-sac, description={passage or street closed at one end}, plural=culs-de-sac}
\newglossaryentry{elite}{name={é}lite, description={select group or class}, sort=elite}
\newglossaryentry{elitism}{name={é}litism, description={advocacy of dominance by an \gls{elite}}, sort=elitism}
\newglossaryentry{attache}{name=attaché, description={person with special diplomatic responsibilities}}

% Eintraege für Symbolliste
\newglossaryentry{ohm}{type=symbols, name={\ensuremath{\Omega}}, sort=Ohm, symbol={\ensuremath{\Omega}}, description={unit of electrical resistance}}
\newglossaryentry{angstrom}{type=symbols, name={\AA}, sort=angström, symbol={\AA}, description={non-SI unit of length}}




%%%%%%%%%%%%%%%% TRIPOD %%%%%%%%%%%%%%%%
\newglossaryentry{tripod}{name=[Trip]od, description={Name der zu entwickelnden Applikation}}
\newglossaryentry{api}{name=API, description={\textit{Application Programming Interface}, Schnittstelle einer Applikation zur Nutzung über andere Programme}}